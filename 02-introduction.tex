% !TEX root = dissertation.tex synopsis.tex

%
% phdtex
%
% Copyright (c) 2014-2018, Andrew Kanner <andrew.kanner@gmail.com>.
% All rights reserved.
%
% SPDX-License-Identifier: CC-BY-4.0
%

\underline{Актуальность темы исследования} \dots{}

Таким образом, сформировалось противоречие между, с одной стороны,
потребностью в \dots{}, а, с другой стороны, \dots{}. В связи с этим
тема данного диссертационного исследования, направленного на разрешение
указанного противоречия путем \dots{}, является
\underline{актуальной}.

\underline{Степень разработанности темы исследования}. В разное время
ощутимый вклад в развитие теории и практики \dots{}, в том числе в
части \dots{} внесли академики \dots{}, а также такие отечественные и
зарубежные ученые, как \dots{}

\dots{} (указать какие вопросы у них не рассмотрены или какие
недостатки имеются) \dots{}

Отсюда вытекает \underline{научная задача} диссертации, суть которой
состоит в \dots{}

\underline{Объектом исследования} данной работы являются \dots{}

\underline{Предметом исследования} являются \dots{}

\underline{Целью} диссертационной работы является \dots{}

\underline{Задачи работы}. Для достижения поставленной цели в работе
решались следующие задачи:
\begin{enumerate}
\item Анализ \dots{}, обоснование постановки задачи исследования;
\item Исследование \dots{};
\item Адаптация существующих или разработка новых \dots{};
\item Проведение экспериментальных исследований разработанной \dots{}.
\end{enumerate}

\underline{Положения, выносимые на защиту}. В диссертационном
исследовании получены и выносятся на защиту следующие положения и
научные результаты:
\begin{enumerate}
\item Предложенный \dots{} устраняет такое-то ограничение \dots{}
  учитывает то-то \dots{} позволяет
  то-то. % (С.~\pageref{sect1_1}-\pageref{sect1_1-eof})
\item Разработанная модель \dots{} (описать процедурную новизну)
  \dots{} расширяет известную модель такую-то тем-то и
  тем-то. % (С.~\pageref{sect2_1}-\pageref{sect2_1-eof})
\end{enumerate}

\underline{Научная новизна работы}. Новизна научных результатов работы
заключается в следующем:
\begin{itemize}
\item Предложены \dots{}, основанные на известных \dots{} и, в отличие
  от них, \dots{} позволяющие устранить выявленные существенные
  недостатки \dots{} за счет введения новых процедур: \dots{}.
\item Предложена модель \dots{}, основанная на известной XX-модели и,
  в отличие от нее, \dots{}, что приводит к тому-то.
\end{itemize}

\underline{Теоретическая значимость работы}. Теоретическая значимость
научных результатов исследования заключается в развитии
теоретико-методологической базы \dots{}, разработке \dots{} широкого
назначения и \dots{} с использованием существующего аппарата теории
\dots{}.

\underline{Практическая ценность работы}. Практическая значимость
результатов исследования заключается в том, что на базе полученных
научных результатов разработана \dots{}, не обладающая недостатками
\dots{}. Самостоятельное практическое значение имеют следующие
результаты работы:
\begin{itemize}
\item Практический результат 1.
\item Практический результат 2.
\item Практический результат 3.
\end{itemize}

\underline{Методы исследования}. В диссертационной работе используются
теория \dots{}. Теоретической основой диссертационного исследования
являются достижения зарубежных и российских ученых и специалистов в
области \dots{}. Информационную базу исследования составляют сведения,
публикуемые в научных изданиях, в периодической печати, материалы
международных и российских научных конференций по рассматриваемой
теме, а также существующие стандарты и требования по \dots{}.

\underline{Соответствие специальности научных работников}. Содержание
диссертационного исследования и полученные научные результаты
соответствуют п. $X$, $XX$ и $XXX$ Паспорта специальности \specnum{}
\specname{}.

\underline{Степень достоверности научных положений и выводов}
обеспечена корректным использованием теорий \dots{}, а также
положительными итогами использования разработанной \dots{} в реальных
проектах и совпадением ожидаемых результатов от использования
предложенных \dots{} при экспериментальных исследованиях.

\underline{Внедрение результатов работы}. Результаты диссертационного
исследования внедрены в \dots{}, используются \dots{} в \dots{}, а
также \dots{} в \dots{} (свидетельство о государственной регистрации
программы для ЭВМ №$XXXXXXXXXX$). Комплекс \dots{}, основанный на
разработанной \dots{}, получил сертификат соответствия требованиям
\dots{} №$XXXX$ по \dots{}. Соответствующие документы, подтверждающие
практическое использование и внедрение результатов исследования,
приведены в приложении к тексту диссертационной работы.

\underline{Апробация работы}. Основные положения и результаты работы
представлялись и обсуждались~на следующих конференциях: \dots{}

\underline{Публикации}. Основные результаты по теме диссертации
изложены в $3$ печатных работах~\cite{article_pub, article_vak,
  article_scopus} общим объемом $X.XX$ п.л., в которых автору
принадлежит $X.XX$ п.л. Из них $1$ печатная работа издана в журнале,
рекомендованном Высшей аттестационной комиссией при Министерстве
образования и науки Российской Федерации~\cite{article_vak}, $1$ -- в
журнале из перечня Scopus \cite{article_scopus}, $1$ --- в тезисе
доклада~\cite{article_pub}.

\underline{Личный вклад}. Содержание диссертации и основные положения,
выносимые на защиту, отражают персональный вклад автора в работу. Все
основные представленные в диссертации результаты получены автором
самостоятельно. В работах, опубликованных в соавторстве, лично автору
принадлежат: \dots{} \cite{article_pub}; \dots{} \cite{article_vak};
\dots{} \cite{article_scopus}.

%\clearpage
