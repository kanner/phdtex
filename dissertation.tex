\documentclass[a4paper,12pt]{report} % добавить leqno в [] для нумерации слева

\IfFileExists{./template/contrib/styles.tex}{\def\template{./template}}{\def\template{.}}

%%% Поля и разметка страницы %%%
\usepackage{lscape}		% Для включения альбомных страниц
\usepackage[headheight=17pt]{geometry}	% Для последующего задания полей (для устранения "\headheight is too small")

%%% Кодировки и шрифты %%%
\usepackage{cmap}						% Улучшенный поиск русских слов в полученном pdf-файле
\usepackage[T2A]{fontenc}				% Поддержка русских букв (кодировка)
\usepackage[utf8]{inputenc}				% Кодировка utf8 (исходного текста)
\usepackage[english, russian]{babel}	% Языки: русский, английский (локализация и переносы)
\usepackage{pscyr}						% Красивые русские шрифты
\usepackage{extsizes}					% Возможность сделать 14-й шрифт (не перенесено в documentclass extreport из-за неподдерживаемости в secsty)

%%% Альтернативные шрифты %%%
%\usepackage[english, russian]{babel}
%\usepackage{fontspec}					% загрузка шрифтов Open Type, True Type и др.
%\usepackage{euscript}					% Шрифт Евклид
%\usepackage{mathrsfs}					% Красивый матшрифт

%%% Оформление абзацев, колонтитулов, текста %%%
\usepackage{indentfirst}	% Красная строка
\frenchspacing
\usepackage{fancyhdr}		% Колонтитулы
\usepackage{setspace}		% Интерлиньяж

%%% Математические пакеты %%%
\usepackage{amsmath,amsfonts,amssymb,amsthm,mathtools,amscd}	% Математические дополнения от AMS
\usepackage{icomma}												% "Умная" запятая: $0,2$ --- число, $0, 2$ --- перечисление
\usepackage{mathtext}											% русские буквы в формулах
%\usepackage{leqno}												% Нумерация формул слева

%%% Цвета %%%
\usepackage[usenames]{color}
\usepackage{color}
\usepackage{colortbl}
\usepackage[usenames,dvipsnames,svgnames,table,rgb]{xcolor}

%%% Таблицы %%%
\usepackage{array,tabularx,tabulary,booktabs}	% Дополнительная работа с таблицами
\usepackage{longtable}							% Длинные таблицы
\usepackage{multirow,makecell}					% Улучшенное форматирование таблиц (слияние строк и т.п.)

%%% Общее форматирование
\usepackage[singlelinecheck=off,center]{caption}	% Многострочные подписи
\usepackage{soul}									% Поддержка переносоустойчивых подчеркиваний и зачеркиваний
%\usepackage{soulutf8}								% Аналогичные модификаторы начертания

%%% Библиография %%%
\usepackage{cite}				% Красивые ссылки на литературу (библиография)
%\usepackage[superscript]{cite}	% Ссылки в верхних индексах
%\usepackage[nocompress]{cite}
\usepackage{csquotes}			% Еще инструменты для ссылок
%\usepackage[backend=biber,bibencoding=utf8,sorting=ynt,maxcitenames=2,style=authoryear]{biblatex}
%\addbibresource{bib1.bib}
%\usepackage[style=authoryear,maxcitenames=2,backend=biber,sorting=nty]{biblatex}

%%% Гиперссылки %%%
\usepackage[linktocpage=true,plainpages=false,pdfpagelabels=false]{hyperref}

%%% Изображения %%%
\usepackage{graphicx}	% Подключаем пакет работы с графикой
\usepackage{wrapfig}	% Обтекание рисунков текстом

%%% Оглавление %%%
\usepackage{tocloft}

%%% Программирование %%%
\usepackage{etoolbox}	% логические операторы

%%% Рисование
\usepackage{tikz}		% Работа с графикой
\usepackage{pgfplots}
\usepackage{pgfplotstable}
\usetikzlibrary{positioning,shapes,shadows,arrows,intersections,patterns}
\pgfplotsset{width=7cm,compat=1.10}
\usepgfplotslibrary{fillbetween}

%%% Другие пакеты %%%
\usepackage{lastpage}	% Узнать, сколько всего страниц в документе.
\usepackage{multicol}	% Несколько колонок
\usepackage{totcount}	% Счетчики для глав, приложений и т.д.
\usepackage{listings}	% Для блоков кода

%%% Выделение текста %%%
\usepackage{varwidth}
\usepackage{xcolor}
\usepackage{lipsum}
% второй вариант
%\usepackage[normalem]{ulem}

%%% Опционально %%%
%\usepackage{savetrees}			% ужимание текста
%\usepackage[print]{booklet}	% печать брошюрой (не тестировалось)

%\usepackage{lmodern} % устранение "Font shape undefined" [http://ctan.org/pkg/lm, http://www.tex.ac.uk/cgi-bin/texfaq2html?label=fontsize]

\usepackage{footmisc}		% Подключаемые пакеты
%%% Колонтитулы (см. packages.tex) %%%
%\pagestyle{fancy}
%\renewcommand{\headrulewidth}{0pt}	% Толщина линейки, отчеркивающей верхний колонтитул
%\lfoot{Нижний левый}
%\rfoot{Нижний правый}
%\rhead{Верхний правый}
%\chead{Верхний в центре}
%\lhead{Верхний левый}
%\cfoot{Нижний в центре}			% По умолчанию здесь номер страницы

%%% Интерлиньяж %%%
\onehalfspacing	% Интерлиньяж 1.5
%\doublespacing	% Интерлиньяж 2
%\singlespacing	% Интерлиньяж 1

%%% Макет страницы %%%
\geometry{a4paper,top=2cm,bottom=2cm,left=3cm,right=1cm}

%%% Кодировки и шрифты %%%
\renewcommand{\rmdefault}{ftm} % Включаем Times New Roman

%%% Альтернативные шрифты (см. packages.tex) %%%
%\defaultfontfeatures{Ligatures={TeX},Renderer=Basic}		% свойства шрифтов по умолчанию
%\setmainfont[Ligatures={TeX,Historic}]{Times New Roman}	% основной шрифт документа
%\setsansfont{Comic Sans MS}								% шрифт без засечек
%\setmonofont{Courier New}
%\renewcommand{\familydefault}{\sfdefault}					% Начертание шрифта

%%% Номера формул %%%
%\mathtoolsset{showonlyrefs=true}	% Показывать номера только у тех формул, на которые есть \eqref{} в тексте.

%%% Выравнивание и переносы %%%
\sloppy					% Избавляемся от переполнений
\clubpenalty=10000		% Запрещаем разрыв страницы после первой строки абзаца
\widowpenalty=10000		% Запрещаем разрыв страницы после последней строки абзаца

%%% Библиография %%%
\makeatletter
\bibliographystyle{./contrib/utf8gost705u}	% Оформляем библиографию в соответствии с ГОСТ 7.0.5
\renewcommand{\@biblabel}[1]{#1.}	% Заменяем библиографию с квадратных скобок на точку:
\makeatother

%%% Изображения %%%
\graphicspath{{images/}}	% Пути к изображениям
\setlength\fboxsep{3pt}		% Отступ рамки \fbox{} от рисунка
\setlength\fboxrule{1pt}	% Толщина линий рамки \fbox{}

%%% Цвета гиперссылок %%%
\definecolor{linkcolor}{rgb}{0.9,0,0}
\definecolor{citecolor}{rgb}{0,0.6,0}
\definecolor{urlcolor}{rgb}{0,0,1}
\hypersetup{				% Гиперссылки
	unicode=true,			% русские буквы в раздела PDF
	pdftitle={title},		% Заголовок
	pdfauthor={author},		% Автор
	pdfsubject={theme},		% Тема
	pdfcreator={creator},	% Создатель
	pdfproducer={producer},	% Производитель
	pdfkeywords={k1} {k2},	% Ключевые слова
    colorlinks=true,		% false: ссылки в рамках; true: цветные ссылки
    linkcolor={linkcolor},	% внутренние ссылки
    citecolor={citecolor},	% на библиографию
    filecolor=magenta,		% на файлы
    urlcolor={urlcolor}		% на URL
}

%%% Оглавление %%%
\renewcommand{\cftchapdotsep}{\cftdotsep}

%%% Теоремы %%%
\theoremstyle{plain}		% Это стиль по умолчанию, его можно не переопределять.
\newtheorem{theorem}{Теорема}[section]
\newtheorem{proposition}[theorem]{Утверждение}
 
\theoremstyle{definition}	% "Определение"
\newtheorem{corollary}{Следствие}[theorem]
\newtheorem{problem}{Задача}[section]
 
\theoremstyle{remark}		% "Примечание"
\newtheorem*{nonum}{Решение}
			% Пользовательские стили
%%% Колонтитулы (см. packages.tex) %%%
\pagestyle{fancy}
\renewcommand{\headrulewidth}{0pt}	% Толщина линейки, отчеркивающей верхний колонтитул
%\lfoot{Нижний левый}
%\rfoot{Нижний правый}
%\rhead{Верхний правый}
%\chead{Верхний в центре}
%\lhead{Верхний левый}
%\cfoot{Нижний в центре}			% По умолчанию здесь номер страницы
\lfoot{}
\rfoot{}
\rhead{}
\chead{\normalsize\thepage}
\lhead{}
\cfoot{}

%%% при использовании chapters - первая страница создается в plain pagestyle %%%
\fancypagestyle{plain}{\pagestyle{fancy}}

%%% Интерлиньяж %%%
\onehalfspacing	% Интерлиньяж 1.5
%\doublespacing	% Интерлиньяж 2
%\singlespacing	% Интерлиньяж 1

%%% Макет страницы %%%
\geometry{a4paper,top=20mm,bottom=20mm,left=25mm,right=10mm}

\usepackage{sectsty}
\allsectionsfont{\centering}	% центрирование заголовков (должны быть без точки на конце и переносов)

%%% список сокращений %%%
\usepackage{nomencl}

% Позволяет одновременно печатать условное обозначение в тексте документа и добавлять его в перечень
\newcommand*{\nom}[2]{#1\nomenclature{#1}{#2}}

%%% Формат формул и ссылок на формулы %%%
% ПРИМЕР:
%\begin{equation}\label{name}
%	2+2=4
%\end{equation}
%
%Формула \eqref{name}

% Перечень сокращений будет распечатываться по алфавиту вне зависимости от появления в тексте
% ПРИМЕР:
%\nom{Б}{Вторая буква алфавита}.
%А\nomenclature{А}{Первая буква алфавита}

% Список литературы формируется в порядке возрастания - bib1, bib2, bib3,...
% по ГОСТ НЕОБХОДИМО, чтобы источники на отличном от русского языке перечислялись ПОСЛЕ источников на русском

% Приложения ДОЛЖНЫ быть перечислены в порядке их перечисления в тексте	% Cтили ГОСТ Р 7.0.11-2011

\usepackage{\template/contrib/tweaklist} % Подключаем пакет tweaklist.sty - формирование отступов в enumerate и itemize

%\includeonly{01-title,\template/contrib/contents,parts/intro,parts/references}

% для проверки вхождения всех источников (+ см. ниже)
%\usepackage{refcheck}

% Сужаем литературу для сохранения места
%\usepackage{etoolbox}
\patchcmd\thebibliography
{\labelsep}
{\labelsep\itemsep=0pt\parsep=0pt\setlength{\leftmargin}{15pt}\relax}
{}
{\typeout{Couldn't patch the command}}

% Переопределяем общие установки для нумерованных перечней - пакет tweaklist.sty
\renewcommand{\enumhook}{
	\setlength{\itemindent}{1.7\parindent}%    Г  1-й абзацный отступ
	\setlength{\itemsep}{0mm}%              В  Разделитель элементов
	\setlength{\labelsep}{.5em}%            Г  Расстояние между маркером и текстом (0,5 кегля)
	\setlength{\labelwidth}{0mm}%           Г  Ширина маркера
	\setlength{\listparindent}{1.5\parindent}% Г  2-й и последующие абзацные отступы
	\setlength{\leftmargin}{0mm}%           Г  Отступ списка слева
	\setlength{\rightmargin}{0mm}%          Г  Отступ списка справа
	\setlength{\parsep}{0mm}%               В  Расстояние между элементами или абзацами
	\setlength{\parskip}{\parskip}%         В  Отступ над вложенным списком
	\setlength{\partopsep}{0mm}%            В  Отступ списка
	\setlength{\topsep}{0mm}%               В  Отступ вложенного списка
}

% Копируем настройки для перечней верхнего уровня
\renewcommand{\enumhooki}{\enumhook}
\renewcommand{\itemhooki}{\enumhook}
\renewcommand{\enumhookii}{\enumhook}
\renewcommand{\itemhookii}{\enumhook}

%%% счетчики списка литературы, количества глав и приложений %%%
\newtotcounter{bibnum}
\newtotcounter{chapnum}
\newtotcounter{appendnum}

% сохраненный object %
\def\oldcite{}
\def\oldchap{}
\def\oldappnd{}
\let\oldcite=\bibcite
\let\oldchap=\chapter
\let\oldappnd=\chapter

% object = object + увеличение счетчика %
\def\bibcite{\stepcounter{bibnum}\oldcite}
\def\realchapter{\stepcounter{chapnum}\oldchap}
\def\appndchapter{\stepcounter{appendnum}\oldappnd}

% !TEX root = dissertation.tex synopsis.tex

%
% phdtex
%
% Copyright (c) 2014-2020, Andrew Kanner <andrew.kanner@gmail.com>.
% All rights reserved.
%
% SPDX-License-Identifier: CC-BY-4.0
%

% заголовок
\author{ФАМИЛИЯ ИМЯ ОТЧЕСТВО автора}
\title{НАЗВАНИЕ ДИССЕРТАЦИОННОЙ РАБОТЫ}

% неизменяемые общие данные
\date{\today}
\makeatletter
\def\dissauthor{\@author}
\def\disstitle{\@title}
\def\dissdate{\@date}
\makeatother
\def\dissyear{\the\year}

% специальность, ученая степень, руководитель, консультант, оппоненты
\def\specnum{ХХ.ХХ.ХХ}
\def\specname{<<Название специальности>>}
\def\edudegree{кандидата каких-то там наук}
\def\mentordegree{уч. степень, уч. звание}
%\def\mentortitle{уч. звание}
\def\mentorcompany{Название\\компании,\\г. Город\\~}
\def\mentorname{Фамилия И. О.}
\def\consultantdegree{уч. степень}
\def\consultantcompany{Название компании,\\ г. Город}
\def\consultantname{Фамилия И. О.}
\def\opponentonedegree{уч. степень}
\def\opponentonecompany{\\Название компании,\\ г. Город}
\def\opponentonename{Фамилия И. О.}
\def\opponenttwodegree{уч. степень}
\def\opponenttwocompany{\\Название компании,\\ г. Город}
\def\opponenttwoname{Фамилия И. О.}
\def\opponentthreedegree{уч. степень}
\def\opponentthreecompany{\\Название компании,\\ г. Город}
\def\opponentthreename{Фамилия И. О.}
\def\leadingorg{Название ведущей организации}

% прочие данные
\def\disscity{Город1 }
\def\disscouncilcity{Город2 }
\def\dissorg{НАЗВАНИЕ УЧРЕЖДЕНИЯ, В КОТОРОМ ВЫПОЛНЯЛАСЬ\par
ДАННАЯ ДИССЕРТАЦИОННАЯ РАБОТА\par}
\def\disscouncil{НАЗВАНИЕ УЧРЕЖДЕНИЯ, В КОТОРОМ ЗАЩИЩАЛАСЬ\par
ДАННАЯ ДИССЕРТАЦИОННАЯ РАБОТА\par}
\def\dissorgsyn{<<Название учреждения, в котором выполнялась данная
  диссертационная работа>>}
\def\dissudk{УДК xxx.xxx}

% данные для оборота титульного листа автореферата
\def\councildate{<<\dots>> \dots \the\year~г.}
\def\counciltime{\dots:00}
\def\councilnum{Д xxx.xxx.xx}
\def\councilplace{<<Организация, к которой относится диссертационный
  совет>>}
\def\counciladdress{Индекс, г.~Город, шоссе / улица / \dots,~д.xx}
\def\libraryname{<<Организация, куда передается рукопись
  диссертационной работы>> и на сайте:}
\def\libraryaddress{Индекс, г.~Город, шоссе / улица / \dots,~д.xx}
\def\councilwebsite{http://website.ru}
\def\synopsissentdate{<<\dots>> \dots \the\year~года}
\def\councilsecretarydegree{уч. степень, уч. звание}
\def\councilsecretaryname{И.О.~Фамилия}

% выставим атрибуты pdf-документа
\hypersetup{
  % заголовок
  pdftitle={\disstitle},
  % автор
  pdfauthor={\dissauthor},
  % тема
  pdfsubject={\disstitle},
  % создатель
  pdfcreator={\dissauthor},
  % производитель
  pdfproducer={\dissauthor},
  % ключевые слова
  pdfkeywords={keyword1} {keyword2}
}


\date{\today}
\makeatletter
\def\dissauthor{\@author}
\def\disstitle{\@title}
\def\dissdate{\@date}
\makeatother
\def\dissyear{\the\year}

%%% выставим атрибуты pdf-документа - см. styles.tex %%%
\hypersetup{
	pdftitle={\disstitle},		% Заголовок
	pdfauthor={\dissauthor},	% Автор
	pdfsubject={\disstitle},	% Тема
	pdfcreator={\dissauthor},	% Создатель
	pdfproducer={\dissauthor},	% Производитель
}

%%% Список сокращений - закомментировать, если не требуется %%%
\makenomenclature

% для переопределения параметров geometry (для правильной работы fancyhdr)
\makeatletter
\newcommand{\resetHeadWidth}{\fancy@setoffs}
\makeatother

\begin{document}

%
% phdtex
%
% Copyright (c) 2014-2018, Andrew Kanner <andrew.kanner@gmail.com>.
% All rights reserved.
%
% SPDX-License-Identifier: MIT
%

% переопределение именований
\renewcommand{\abstractname}{Аннотация}
\renewcommand{\alsoname}{см. также}
\renewcommand{\appendixname}{Приложение}
\renewcommand{\bibname}{Список литературы}
\renewcommand{\ccname}{исх.}
\renewcommand{\chaptername}{Глава}
\renewcommand{\contentsname}{Оглавление}
\renewcommand{\enclname}{вкл.}
\renewcommand{\figurename}{Рисунок}
\renewcommand{\headtoname}{вх.}
\renewcommand{\indexname}{Предметный указатель}
\renewcommand{\listfigurename}{Список рисунков}
\renewcommand{\listtablename}{Список таблиц}
\renewcommand{\pagename}{Стр.}
\renewcommand{\partname}{Часть}
\renewcommand{\refname}{Список литературы}
\renewcommand{\seename}{см.}
\renewcommand{\tablename}{Таблица}
\renewcommand{\lstlistingname}{Листинг}

% заголовок перечня nomenclature
\renewcommand{\nomname}{Список сокращений и условных обозначений}

% переопределение математических символов на русский манер
\renewcommand{\epsilon}{\ensuremath{\varepsilon}}
\renewcommand{\phi}{\ensuremath{\varphi}}
\renewcommand{\kappa}{\ensuremath{\varkappa}}
\renewcommand{\le}{\ensuremath{\leqslant}}
\renewcommand{\leq}{\ensuremath{\leqslant}}
\renewcommand{\ge}{\ensuremath{\geqslant}}
\renewcommand{\geq}{\ensuremath{\geqslant}}
\renewcommand{\emptyset}{\varnothing}

% перенос знаков в формулах (по Львовскому)
\newcommand*{\hm}[1]{#1\nobreak\discretionary{}
{\hbox{$\mathsurround=0pt #1$}}{}}

% сокращения
\newcommand*{\linux}{GNU/Linux}
\newcommand*{\linuxkernel}{Linux}

% дополнительные команды
%\DeclareMathOperator{\sgn}{\mathop{sgn}}
\newcommand*{\nbline}{\\*\indent}
\newcommand{\tab}[1]{\hspace{.1\textwidth}\rlap{#1}}
\newcommand*{\ti}[1]{\text{\textit{#1}}}

% счетчик для пунктов в первом столбце таблиц
\newcounter{rowcount}
\setcounter{rowcount}{0}
\newcommand\rownumber{\stepcounter{rowcount}\arabic{rowcount}.}
			% Переопределение именований

% !TEX root = dissertation.tex

\thispagestyle{empty}

\begin{center}
\dissorg
\par
\end{center}

\vspace{20mm}
\begin{flushright}
На правах рукописи

{\sl \dissudk}
\end{flushright}

\vspace{30mm}
\begin{center}
{\large \dissauthor}
\end{center}

\vspace{5mm}
\begin{center}
{\bf \large \disstitle
\par}

\vspace{10mm}
{%\small
Специальность \specnum~---

\specname
}

\vspace{10mm}
Диссертация на соискание учёной степени

\edudegree
\end{center}

\vspace{20mm}
\begin{flushright}
Научный руководитель:

\mentordegree

\mentorname

\end{flushright}

\vspace{20mm}
\begin{center}
{\disscity -- \dissyear}
\end{center}

%\newpage			% Титульный лист
\resetHeadWidth
% !TEX root = ../dissertation.tex

\tableofcontents
%\clearpage
		% Оглавление
% Части Введения для возможности исключения в \includeonly (основной текст в introduction.tex)
% !TEX root = ../dissertation.tex

% вынесено из 02-introduction.tex с целью использования разных
% заголовков в автореферате и тексте диссертации

% заголовок
\begin{singlespace}
  \chapter*{\MakeUppercase{Введение}}
\end{singlespace}
% добавляем введение в оглавление
\addcontentsline{toc}{chapter}{Введение}

% введение
% !TEX root = dissertation.tex

\chapter*{Введение}

% добавляем введение в оглавление
\addcontentsline{toc}{chapter}{Введение}

Текст введения

%\clearpage
		% Введение
% !TEX root = dissertation.tex

%
% phdtex
%
% Copyright (c) 2014-2017, Andrew Kanner <andrew.kanner@gmail.com>.
% All rights reserved.
%
% SPDX-License-Identifier: CC-BY-4.0
%

\begin{singlespace}
  \realchapter{Название главы 1} \label{chapt1}
\end{singlespace}

% комментарии разработчика шаблона
Для более удобного использования cvs типа git лучше использовать
редактор с переносом строк длиной более 80 символов. Так будет проще
вносить и затем просматривать изменения по длинным абзацам.

Перенос строк в latex не является признаком начала нового абзаца,
новый абзац начинается если перед текстом вставлена пустая строка (как
перед этим предложением). \\

\section{Название раздела 1.1} \label{sect1_1}

Текст раздела \dots{} \\

Нумерованный список:
\begin{enumerate}
\item пункт 1;
\item пункт 2;
\item пункт 3. \\
\end{enumerate}

Ненумерованный список:
\begin{itemize}
\item пункт 1;
\item пункт 2;
\item пункт 3. \\
\end{itemize}

Список c произвольными разделителями:
\begin{description}
\item{\textbf{[п1]}} пункт 1;
\item{\textbf{[п2]}} пункт 2;
\item{\textbf{[п3]}} пункт 3. \\
\end{description}

Совмещенные списки:
\begin{enumerate}
\item пункт 1:
      \begin{itemize}
      \item подпункт 1.
      \end{itemize}
\item пункт 2;
\item пункт 3. \\
\end{enumerate}

Список myenumerate (см. contrib/stylesgost.tex):
\begin{myenumerate}
\item пункт 1;

  Дополнительный длинный-длинный-длинный текст без изменения
  форматирования по отношению к основному тексту вне списка.

\item пункт 2;
\item пункт 3. \\
\end{myenumerate}

Пример рисунка (см. Рисунок \ref{test1}). \\

\begin{figure}[bhtp]
  \centering \includegraphics[scale=0.1]{latex-cc0}
  \caption{Подпись рисунка} \label{test1}
\end{figure}

Символ\_подчеркивания, символ процента -- \%. \\

Числа вводятся в math mode: $1$, $2$, $3$ \dots{} \\

Запрет переноса специфических терминов, неразрывный пробел:
\mbox{идентификация/аутентификация} или \mbox{ФСТЭК},
эту~фразу~переносить~нельзя
(только~целиком~или~по~правилам~переносов~слов). \\

Сокращения и условные обозначения: операционная система
(\nom{ОС}{Операционная система}). \\

<<Правильные>> кавычки. Правильное тире --, теперь длинное ---. \\

Выделения текста: \textbf{bold}, \textit{italic},
\underline{underline}, \hly{цветом}, \dots{} \\

Ссылки: литературные источники \cite{article_other}, несколько вместе
\cite{article_pub, article_vak, article_scopus, book1, thesis1,
  conference1, hyperlink1}; сноски\footnote{Текст сноски.}; ссылки на
любые помеченные сущности с помощью label, например на главу
\ref{chapt1}, ссылки на страницу с помеченной сущностью
\pageref{chapt1}. \\

\subsection{Подраздел} \label{subsect1_1_1}

\subsubsection{<<Подподраздел>>} \label{subsect1_1_1_1}

% этот label используется для простановки номеров страниц в Положениях
% из введения
Конец раздела.\label{sect1_1-eof}

%\newpage
%===============================================================================

\section{Название раздела 1.2} \label{sect1_2}

%\newpage
%===============================================================================

\section{Выводы} \label{sect1_3}

\begin{enumerate}
\item Вывод 1.
\item Вывод 2.
\item Вывод 3.
\end{enumerate}

%\clearpage
			% Глава 1
% !TEX root = dissertation.tex

%
% phdtex
%
% Copyright (c) 2014-2017, Andrew Kanner <andrew.kanner@gmail.com>.
% All rights reserved.
%
% SPDX-License-Identifier: CC-BY-4.0
%

\begin{singlespace}
  \realchapter{Название главы 2} \label{chapt2}
\end{singlespace}

\section{Название раздела 2.1} \label{sect2_1}

Разные формулы:

Индекс -- $t_i$.

Составной индекс -- $t_{i+1}$.

Верхний и верхний составной индексы -- $t^i$, $t^{i+1}$.

Различные математические символы: $=$ $\leq$ $\geq$ $\neq$ $\subseteq$
$\cup$ $\cap$ $\rightarrow$ $\in$ $\emptyset$ $\forall$ $\sin$ $\cos$
$\tg$ $\ctg$ $\ln$ $\text{текст в math mode}$ и так далее.

Скобки с автоувеличением: $\left(\frac{a}{b}\right)$,
$\left\{A,B\right\}$,

$\alpha$, $\Phi$, $\epsilon$, $\kappa$ \dots{}

Диакритические знаки: $\bar x = 1$, $\overline{xy} = 1$,
$\tilde x = 1$, $\widetilde{xy} = 1$

Выключная формула:
\[
  a + b = c
\]

Нумерованная выключная формула \refeq{eq1}:
\begin{equation}\label{eq1}
  a + b = c
\end{equation}

Интеграл: $\int_0^1 x = 1/2$

Умножение: $2 \times 2 = 4$, $2^2 = 4$

Дроби: $\frac{5}{6}=\frac{1}{1+\frac{1}{5}}$

Формула в несколько строк:
\begin{multline}
  1 + 2 + 3 \dots + \\ + 50 + 51 + 52 + \dots + \\ + 99 + 100 = 5050
\end{multline}

Система уравнений:
\[\left\{
    \begin{aligned}
      x+1=y \\
      y=x-1
    \end{aligned}
  \right.
\]

Вариативность:
\[
  |x| = \begin{cases}
    x, &\text{если \dots} \\
    x+1, &\text{иначе}
  \end{cases}
\]

Матрицы:
\[
  \begin{pmatrix}
    a_{11} & a_{12} & a_{13} \\
    \dots
  \end{pmatrix}
\]

И так далее.

\begin{definition} \label{definition1} Текст определения.
\end{definition}

\begin{proposition} \label{proposition1} Текст утверждения.
\end{proposition}

\begin{lemma} \label{lemma1} Текст леммы.
\end{lemma}

\begin{theorem}[Название теоремы] \label{theorem1} Текст теоремы.
\end{theorem}

\begin{proof}
  Доказательство \dots{}
\end{proof}

\begin{corollary} \label{corollary1} Текст следствия.
\end{corollary}

% этот label используется для простановки номеров страниц в Положениях
% из введения
Конец раздела.\label{sect2_1-eof}

%\newpage
%===============================================================================

\section{Выводы} \label{sect2_2}

\begin{enumerate}
\item Вывод 1.
\item Вывод 2.
\item Вывод 3.
\end{enumerate}

%\clearpage
			% Глава 2
% !TEX root = dissertation.tex

\begin{singlespace}
  \realchapter{Название главы 3} \label{chapt3}
\end{singlespace}

\section{Название раздела 3.1} \label{sect3_1}

Какой-нибудь короткий листинг, см. листинг \ref{listing1} (ссылка на
длинный листинг в Приложении~\ref{AppendixA}) и другие материалы по
разработанному (внедренному куда-нибудь) на базе полученных ранее
научных результатов.

\begin{lstlisting}[language=C,label=listing1,caption=Пример короткого
  листинга на языке C (\textbf{blank.c})]
  int main(void) {
    return 0;
  }
\end{lstlisting}

%\newpage
%===============================================================================

\section{Выводы} \label{sect3_2}

\begin{enumerate}
\item Вывод 1.
\item Вывод 2.
\item Вывод 3.
\end{enumerate}

%\clearpage
			% Глава 3
% !TEX root = dissertation.tex

%
% phdtex
%
% Copyright (c) 2014-2017, Andrew Kanner <andrew.kanner@gmail.com>.
% All rights reserved.
%
% SPDX-License-Identifier: CC-BY-4.0
%

\begin{singlespace}
  \realchapter{Название главы 4} \label{chapt4}
\end{singlespace}

\section{Название раздела 4.1} \label{sect4_1}

Какой-нибудь график и другие результаты экспериментальных исследований
(см. рисунок~\ref{test-tikz}).

\begin{figure}[h!]
  \centering
  \begin{tikzpicture}
    \begin{axis}[width=0.6\textwidth, height=0.7\textwidth, xmin=0,
      xmax=6, ymin=0, ymax=100, ytick={0,100}, yticklabels={0,100\%},
      xtick={0,1,2,3,4,5,6}, xticklabels={, $2012$, $2013$, $2014$,
        $2015$, $2016$, гг.},]

      \addplot[name path=axs1, white] coordinates {(1,99)
        (5,99)};

      \addplot[name path=plt1,red!80,smooth,ultra thick] coordinates {
        (1,50) (2,60) (3,75) (4,85) (5,99) };

      \addplot[black] fill between[of=plt1 and axs1, split, every
      segment no 0/.style={pattern color=red!20, pattern=north east
        lines}, every segment no 1/.style={pattern color=red!20,
        pattern=north east lines}, ];

      \end{axis}
    \end{tikzpicture}

    \caption{Подпись к графику}
      \label{test-tikz}

\end{figure}

%\newpage
%===============================================================================

\section{Название раздела 4.2} \label{sect4_2}

Про апробацию и внедрение, покрытие статьями текста диссертации,
личный вклад автора \dots{}

Возможные направления дальнейших исследований по данной теме.

%\newpage
%===============================================================================

\section{Выводы} \label{sect4_3}

\begin{enumerate}
\item Вывод 1.
\item Вывод 2.
\item Вывод 3.
\end{enumerate}

%\clearpage
			% Глава 4
% !TEX root = dissertation.tex

\begin{singlespace}
  \chapter*{\MakeUppercase{Заключение}}
\end{singlespace}

% добавляем заключение в оглавление
\addcontentsline{toc}{chapter}{Заключение}

Основные результаты работы, заключаются в следующем:

\begin{enumerate}
\item Проведен анализ \dots{} исследовано \dots{} предложено \dots{}
\item Сформированы \dots{}.
\item На базе полученных научных результатов разработано \dots{}
\item В ходе экспериментальных исследований подтвержден ожидаемый
  эффект от применения \dots{}, а также соответствие \dots{}.
\end{enumerate}

%\clearpage
		% Заключение
% !TEX root = ../dissertation.tex

%
% phdtex
%
% Copyright (c) 2014-2018, Andrew Kanner <andrew.kanner@gmail.com>.
% All rights reserved.
%
% SPDX-License-Identifier: MIT
%

\clearpage
\phantomsection \addcontentsline{toc}{chapter}{\nomname}
% список сокращений и условных обозначений
\printnomenclature
\newpage

% список литературы
% !TEX root = ../dissertation.tex

%
% phdtex
%
% Copyright (c) 2014-2017, Andrew Kanner <andrew.kanner@gmail.com>.
% All rights reserved.
%
% SPDX-License-Identifier: MIT
%

\clearpage
\phantomsection
% добавляем список литературы в оглавление
\addcontentsline{toc}{chapter}{\bibname}
% подключаем bibtex-файлы
\bibliography{parts/biblio,parts/biblio-pub,parts/biblio-vak}

% раскомментировать для пометки не вошедших источников из bibtex-баз
%\nocite{*}

% список иллюстративного материала (ГОСТ Р 7.0.11-2011, стр.2)
\clearpage
\phantomsection \addcontentsline{toc}{chapter}{Список иллюстративного материала}
\begin{singlespace}
  \realchapter*{Список иллюстративного материала}
\end{singlespace}

% список изображений
\listoffigures
% сгруппировано, чтобы не создавать на новой странице
\begingroup
\let\clearpage\relax
% cписок таблиц
\listoftables
\endgroup
\newpage

%\clearpage
%\phantomsection \addcontentsline{toc}{chapter}{\listfigurename}
%% список изображений
%\listoffigures
%\newpage

%\clearpage
%\phantomsection \addcontentsline{toc}{chapter}{\listtablename}
%% cписок таблиц
%\listoftables
%\newpage
			% Списки таблиц и изображений
% !TEX root = ../dissertation.tex

%
% phdtex
%
% Copyright (c) 2014-2017, Andrew Kanner <andrew.kanner@gmail.com>.
% All rights reserved.
%
% SPDX-License-Identifier: MIT
%

\clearpage
\phantomsection
% добавляем список литературы в оглавление
\addcontentsline{toc}{chapter}{\bibname}
% подключаем bibtex-файлы
\bibliography{parts/biblio,parts/biblio-pub,parts/biblio-vak}
		% Список литературы
% !TEX root = ../dissertation.tex

\appendix
\appndchapter{Название первого приложения} \label{AppendixA}

Текст приложения

%\clearpage		% Приложения

%%% раскомментировать для пометки не вошедших источников из bibtex баз
%\nocite{*}

\end{document}
