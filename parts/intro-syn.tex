% !TEX root = ../synopsis.tex

%
% phdtex
%
% Copyright (c) 2014-2020, Andrew Kanner <andrew.kanner@gmail.com>.
% All rights reserved.
%
% SPDX-License-Identifier: CC-BY-4.0
%

% вынесено из 02-introduction.tex с целью: 1. использования разных
% заголовков в автореферате и тексте диссертации 2. добавления в
% автореферат блока со структурой работы

% нумерация с 3й страницы
\setcounter{page}{3}

% заголовок
\subsection*{\MakeUppercase{Общая характеристика работы}}

% введение
% !TEX root = dissertation.tex

\chapter*{Введение}

% добавляем введение в оглавление
\addcontentsline{toc}{chapter}{Введение}

Текст введения

%\clearpage

% автоматически проставим значения счетчиков из текста диссертации
\getcounter{mychapters}
\getcounter{myappendices}
\getcounter{mypages}
\getcounter{myfigures}
\getcounter{mytables}
\getcounter{mybibitems}

\underline{Структура и объем работы}. Диссертационная работа состоит
из~введения, \themychapters~глав, заключения
и \themyappendices~приложений. Объем основного текста диссертации
составляет \themypages~страниц(ы) с \themyfigures~рисунками
и \themytables~таблицами. Приложение также содержит документы,
подтверждающие практическое использование и внедрение результатов
диссертационного исследования. Количество наименований в списке
литературы --- \themybibitems.
