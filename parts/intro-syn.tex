% !TEX root = ../synopsis.tex

%
% phdtex
%
% Copyright (c) 2014-2017, Andrew Kanner <andrew.kanner@gmail.com>.
% All rights reserved.
%
% SPDX-License-Identifier: CC-BY-4.0
%

% вынесено из 02-introduction.tex с целью: 1. использования разных
% заголовков в автореферате и тексте диссертации 2. добавления в
% автореферат блока со структурой работы

% нумерация с 3й страницы
\setcounter{page}{3}

% заголовок
\subsection*{\MakeUppercase{Общая характеристика работы}}

% введение
% !TEX root = dissertation.tex

\chapter*{Введение}

% добавляем введение в оглавление
\addcontentsline{toc}{chapter}{Введение}

Текст введения

%\clearpage

% автоматически проставить значения переменных как в тексте
% диссертации тут нельзя -- обновлять нужно вручную (в тексте --
% \total{chapnum}, \total{appendnum}, \pageref{LastPage},
% \total{bibnum} и другие)
\underline{Структура и объем работы}. Диссертационная работа состоит
из~введения, \hly{X} глав, заключения и~\hly{X} приложений. Объем
основного текста диссертации составляет \hly{XXX}~страниц
с~\hly{XX}~рисунками и~\hly{XX}~таблицами. Приложение также содержит
документы, подтверждающие практическое использование и внедрение
результатов диссертационного исследования. Количество наименований в
списке литературы --- \hly{XXX}.
