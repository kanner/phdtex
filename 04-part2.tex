% !TEX root = dissertation.tex

%
% phdtex
%
% Copyright (c) 2014-2017, Andrew Kanner <andrew.kanner@gmail.com>.
% All rights reserved.
%
% SPDX-License-Identifier: CC-BY-4.0
%

\begin{singlespace}
  \realchapter{Название главы 2} \label{chapt2}
\end{singlespace}

\section{Название раздела 2.1} \label{sect2_1}

Разные формулы:

Индекс -- $t_i$.

Составной индекс -- $t_{i+1}$.

Верхний и верхний составной индексы -- $t^i$, $t^{i+1}$.

Различные математические символы: $=$ $\leq$ $\geq$ $\neq$ $\subseteq$
$\cup$ $\cap$ $\rightarrow$ $\in$ $\emptyset$ $\forall$ $\sin$ $\cos$
$\tg$ $\ctg$ $\ln$ $\text{текст в math mode}$ и так далее.

Скобки с автоувеличением: $\left(\frac{a}{b}\right)$,
$\left\{A,B\right\}$,

$\alpha$, $\Phi$, $\epsilon$, $\kappa$ \dots{}

Диакритические знаки: $\bar x = 1$, $\overline{xy} = 1$,
$\tilde x = 1$, $\widetilde{xy} = 1$

Выключная формула:
\[
  a + b = c
\]

Нумерованная выключная формула \refeq{eq1}:
\begin{equation}\label{eq1}
  a + b = c
\end{equation}

Интеграл: $\int_0^1 x = 1/2$

Умножение: $2 \times 2 = 4$, $2^2 = 4$

Дроби: $\frac{5}{6}=\frac{1}{1+\frac{1}{5}}$

Формула в несколько строк:
\begin{multline}
  1 + 2 + 3 \dots + \\ + 50 + 51 + 52 + \dots + \\ + 99 + 100 = 5050
\end{multline}

Система уравнений:
\[\left\{
    \begin{aligned}
      x+1=y \\
      y=x-1
    \end{aligned}
  \right.
\]

Вариативность:
\[
  |x| = \begin{cases}
    x, &\text{если \dots} \\
    x+1, &\text{иначе}
  \end{cases}
\]

Матрицы:
\[
  \begin{pmatrix}
    a_{11} & a_{12} & a_{13} \\
    \dots
  \end{pmatrix}
\]

И так далее.

\begin{definition} \label{definition1} Текст определения.
\end{definition}

\begin{proposition} \label{proposition1} Текст утверждения.
\end{proposition}

\begin{lemma} \label{lemma1} Текст леммы.
\end{lemma}

\begin{theorem}[Название теоремы] \label{theorem1} Текст теоремы.
\end{theorem}

\begin{proof}
  Доказательство \dots{}
\end{proof}

\begin{corollary} \label{corollary1} Текст следствия.
\end{corollary}

% этот label используется для простановки номеров страниц в Положениях
% из введения
Конец раздела.\label{sect2_1-eof}

%\newpage
%===============================================================================

\section{Выводы} \label{sect2_2}

\begin{enumerate}
\item Вывод 1.
\item Вывод 2.
\item Вывод 3.
\end{enumerate}

%\clearpage
