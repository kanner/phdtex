% !TEX root = ../synopsis.tex

%
% phdtex
%
% Copyright (c) 2014-2017, Andrew Kanner <andrew.kanner@gmail.com>.
% All rights reserved.
%
% SPDX-License-Identifier: CC-BY-4.0
%

\subsection*{\MakeUppercase{Основное содержание работы}}

Во \underline{введении} обосновывается актуальность темы
диссертационного исследования, формулируются его цель и задачи,
определяются научная новизна, теоретическая и практическая значимость
полученных результатов. Рассматриваются положения, выносимые на
защиту, апробация и внедрение результатов.

% использовать следующие слова: рассматривается, проводится, показано,
% обосновано, проведен анализ

% \newpage
% ====================================================================

\underline{В первой главе} приводится \dots{}

\dots{}

Таким образом, в главе обоснована актуальность темы диссертационного
исследования, сформулированы постановка научной задачи и возникшее
противоречие, для разрешения которого намечены возможные пути решения
научной задачи исследования и определен состав необходимого
методического обеспечения \dots{}

% \newpage
% ====================================================================

\underline{Вторая глава} посвящена \dots{}

\dots{} какая-нибудь умная формула / график / что-то еще (см. формулу
\ref{test1}) \dots{}

\begin{center}
\begin{equation} \label{test1}
\frac{1}{\Bigl(\sqrt{\phi \sqrt{5}}-\phi\Bigr) e^{\frac25 \pi}} =
1+\frac{e^{-2\pi}} {1+\frac{e^{-4\pi}} {1+\frac{e^{-6\pi}}
{1+\frac{e^{-8\pi}} {1+\ldots} } } }
\end{equation}
\end{center}

Вывод по второй главе.

% \newpage
% ====================================================================

\underline{Третья глава} работы посвящена \dots{}

\dots{} какая-нибудь таблица (см. таблицу \ref{test2}) \dots{}

\setcounter{rowcount}{0}
%\captionsetup{belowskip=-10pt}
\begin{longtable}{|p{0.2\linewidth}|c|}
  \caption{Подпись таблицы -- сверху} \label{test2} \\

  \hline

  \begin{tabular}{@{}l@{}}\textbf{Текст1} \\
                 \textbf{Текст2}\end{tabular} & \textbf{Текст3} \\

  \hline
  \endfirsthead

  \multicolumn{2}{l}{\small Продолжение таблицы \ref{test2}} \\
  \hline

  \begin{tabular}{@{}l@{}}\textbf{Текст1} \\
                 \textbf{Текст2}\end{tabular} & \textbf{Текст3} \\

  \hline
  \endhead %\hline

  \multicolumn{2}{|r|}{} \\

  \endfoot %\hline
  \endlastfoot

  \rownumber{} \cellcolor{green!25}Отдельный & \cellcolor{red!25}столбец \\
  \hline

  \rownumber{} \cellcolor{yellow!25}с нумерацией & \cellcolor{green!25}делать \\
  \hline

  \rownumber{} \cellcolor{red!25}нельзя, см. & \cellcolor{yellow!25}ГОСТ 2.105-95 \\
  \hline

\end{longtable}

Вывод по третьей главе.

% \newpage
% ====================================================================

В \underline{четвертой главе} проводится исследование \dots{}

Вывод по четвертой главе.

% \newpage
% ====================================================================
