% !TEX root = dissertation.tex synopsis.tex

%
% phdtex
%
% Copyright (c) 2014-2017, Andrew Kanner <andrew.kanner@gmail.com>.
% All rights reserved.
%
% SPDX-License-Identifier: CC-BY-4.0
%

% заголовок
\author{ФАМИЛИЯ ИМЯ ОТЧЕСТВО автора}
\title{НАЗВАНИЕ ДИССЕРТАЦИОННОЙ РАБОТЫ}

% неизменяемые общие данные
\date{\today}
\makeatletter
\def\dissauthor{\@author}
\def\disstitle{\@title}
\def\dissdate{\@date}
\makeatother
\def\dissyear{\the\year}

% специальность, ученая степень, руководитель, консультант, оппоненты
\def\specnum{ХХ.ХХ.ХХ}
\def\specname{<<Название специальности>>}
\def\edudegree{кандидата каких-то там наук}
\def\mentordegree{уч. степень, уч. звание}
\def\mentorcompany{Название компании, г. Город}
\def\mentorname{Фамилия И. О.}
\def\consultantdegree{уч. степень, уч. звание}
\def\consultantcompany{Название компании, г. Город}
\def\consultantname{Фамилия И. О.}
\def\opponentonedegree{уч. степень, уч. звание}
\def\opponentonecompany{Название компании, г. Город}
\def\opponentonename{Фамилия И. О.}
\def\opponenttwodegree{уч. степень, уч. звание}
\def\opponenttwocompany{Название компании, г. Город}
\def\opponenttwoname{Фамилия И. О.}

% прочие данные
\def\disscity{Город }
\def\dissorg{НАЗВАНИЕ УЧРЕЖДЕНИЯ, В КОТОРОМ ВЫПОЛНЯЛАСЬ\par
ДАННАЯ ДИССЕРТАЦИОННАЯ РАБОТА\par}
\def\dissorgsyn{<<Название учреждения, в котором выполнялась данная
  диссертационная работа>>}
\def\dissudk{УДК xxx.xxx}

% данные для оборота титульного листа автореферата
\def\councildate{<<\dots>> \dots \the\year~г.}
\def\counciltime{\dots:00}
\def\councilnum{Д xxx.xxx.xx}
\def\councilplace{<<Организация, к которой относится диссертационный
  совет>>}
\def\counciladdress{Индекс, г.~Город, шоссе / улица / \dots,~д.xx}
\def\libraryname{<<Организация, куда передается рукопись
  диссертационной работы>>}
\def\libraryaddress{Индекс, г.~Город, шоссе / улица / \dots,~д.xx}
\def\councilwebsite{http://website.ru}
\def\synopsissentdate{<<\dots>> \dots \the\year~года}
\def\councilsecretarydegree{уч. степень, уч. звание}
\def\councilsecretaryname{И.О.~Фамилия}

% выставим атрибуты pdf-документа
\hypersetup{
  % заголовок
  pdftitle={\disstitle},
  % автор
  pdfauthor={\dissauthor},
  % тема
  pdfsubject={\disstitle},
  % создатель
  pdfcreator={\dissauthor},
  % производитель
  pdfproducer={\dissauthor},
  % ключевые слова
  pdfkeywords={keyword1} {keyword2}
}
