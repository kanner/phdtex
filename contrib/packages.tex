%%% Поля и разметка страницы %%%
\usepackage{lscape}		% Для включения альбомных страниц
\usepackage[headheight=17pt]{geometry}	% Для последующего задания полей (для устранения "\headheight is too small")

%%% Кодировки и шрифты %%%
\usepackage{cmap}						% Улучшенный поиск русских слов в полученном pdf-файле
\usepackage[T2A]{fontenc}				% Поддержка русских букв (кодировка)
\usepackage[utf8]{inputenc}				% Кодировка utf8 (исходного текста)
\usepackage[english, russian]{babel}	% Языки: русский, английский (локализация и переносы)
\usepackage{pscyr}						% Красивые русские шрифты
\usepackage{extsizes}					% Возможность сделать 14-й шрифт (не перенесено в documentclass extreport из-за неподдерживаемости в secsty)

%%% Альтернативные шрифты %%%
%\usepackage[english, russian]{babel}
%\usepackage{fontspec}					% загрузка шрифтов Open Type, True Type и др.
%\usepackage{euscript}					% Шрифт Евклид
%\usepackage{mathrsfs}					% Красивый матшрифт

%%% Оформление абзацев, колонтитулов, текста %%%
\usepackage{indentfirst}	% Красная строка
\frenchspacing
\usepackage{fancyhdr}		% Колонтитулы
\usepackage{setspace}		% Интерлиньяж

%%% Математические пакеты %%%
\usepackage{amsmath,amsfonts,amssymb,amsthm,mathtools,amscd}	% Математические дополнения от AMS
\usepackage{icomma}												% "Умная" запятая: $0,2$ --- число, $0, 2$ --- перечисление
\usepackage{mathtext}											% русские буквы в формулах
%\usepackage{leqno}												% Нумерация формул слева

%%% Цвета %%%
\usepackage[usenames]{color}
\usepackage{color}
\usepackage{colortbl}
\usepackage[usenames,dvipsnames,svgnames,table,rgb]{xcolor}

%%% Таблицы %%%
\usepackage{array,tabularx,tabulary,booktabs}	% Дополнительная работа с таблицами
\usepackage{longtable}							% Длинные таблицы
\usepackage{multirow,makecell}					% Улучшенное форматирование таблиц (слияние строк и т.п.)

%%% Общее форматирование
\usepackage[singlelinecheck=off,center]{caption}	% Многострочные подписи
\usepackage{soul}									% Поддержка переносоустойчивых подчеркиваний и зачеркиваний
%\usepackage{soulutf8}								% Аналогичные модификаторы начертания

%%% Библиография %%%
\usepackage{cite}				% Красивые ссылки на литературу (библиография)
%\usepackage[superscript]{cite}	% Ссылки в верхних индексах
%\usepackage[nocompress]{cite}
\usepackage{csquotes}			% Еще инструменты для ссылок
%\usepackage[backend=biber,bibencoding=utf8,sorting=ynt,maxcitenames=2,style=authoryear]{biblatex}
%\addbibresource{bib1.bib}
%\usepackage[style=authoryear,maxcitenames=2,backend=biber,sorting=nty]{biblatex}

%%% Гиперссылки %%%
\usepackage[linktocpage=true,plainpages=false,pdfpagelabels=false]{hyperref}

%%% Изображения %%%
\usepackage{graphicx}	% Подключаем пакет работы с графикой
\usepackage{wrapfig}	% Обтекание рисунков текстом

%%% Оглавление %%%
\usepackage{tocloft}

%%% Программирование %%%
\usepackage{etoolbox}	% логические операторы

%%% Рисование
\usepackage{tikz}		% Работа с графикой
\usepackage{pgfplots}
\usepackage{pgfplotstable}
\usetikzlibrary{positioning,shapes,shadows,arrows,intersections,patterns}
\pgfplotsset{width=7cm,compat=1.10}
\usepgfplotslibrary{fillbetween}

%%% Другие пакеты %%%
\usepackage{lastpage}	% Узнать, сколько всего страниц в документе.
\usepackage{multicol}	% Несколько колонок
\usepackage{totcount}	% Счетчики для глав, приложений и т.д.
\usepackage{listings}	% Для блоков кода

%%% Выделение текста %%%
\usepackage{varwidth}
\usepackage{xcolor}
\usepackage{lipsum}
% второй вариант
%\usepackage[normalem]{ulem}

%%% Опционально %%%
%\usepackage{savetrees}			% ужимание текста
%\usepackage[print]{booklet}	% печать брошюрой (не тестировалось)

%\usepackage{lmodern} % устранение "Font shape undefined" [http://ctan.org/pkg/lm, http://www.tex.ac.uk/cgi-bin/texfaq2html?label=fontsize]

\usepackage{footmisc}