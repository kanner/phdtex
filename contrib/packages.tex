%
% phdtex
%
% Copyright (c) 2014-2017, Andrew Kanner <andrew.kanner@gmail.com>.
% All rights reserved.
%
% SPDX-License-Identifier: MIT
%

%% Поля и разметка страницы
% для альбомных страниц
\usepackage{lscape}
% для задания полей (+ устранение "\headheight is too small")
\usepackage[headheight=17pt]{geometry}

%% Кодировки и шрифты
% улучшенный поиск русских слов в pdf
\usepackage{cmap}
% поддержка русских букв (кодировка)
\usepackage[T2A]{fontenc}
% кодировка исходного текста
\usepackage[utf8]{inputenc}
% языки, локализация и переносы
\usepackage[english, russian]{babel}
% красивые шрифты pscyr
\usepackage{pscyr}
% добавляет 14й шрифт (не перенесено в documentclass extreport из-за
% несовместимости с secsty)
\usepackage{extsizes}
% альтернативные шрифты
%\usepackage[english, russian]{babel}
% шрифты {open,true}type и др.
%\usepackage{fontspec}
% шрифт Евклид
%\usepackage{euscript}
% математический шрифт
%\usepackage{mathrsfs}

%% Оформление абзацев, колонтитулов, текста
% красная строка
\usepackage{indentfirst}
\frenchspacing
% колонтитулы
\usepackage{fancyhdr}
% интерлиньяж
\usepackage{setspace}
% многострочные подписи
\usepackage[singlelinecheck=off,center]{caption}
% переносоустойчивые под-/зачеркивания (модификаторы начертания)
\usepackage{soul}
%\usepackage{soulutf8}

%% Для работы с формулами
% дополнения от AMS
\usepackage{amsmath,amsfonts,amssymb,amsthm,mathtools,amscd}
% правила для запятой: $0,1$ -- число, $0, 1$ -- перечисление
\usepackage{icomma}
% текст в формулах
\usepackage{mathtext}
% для нумерации формул слева (см. комментарии к documentclass)
%\usepackage{leqno}

%% Цветовые эффекты
\usepackage[usenames]{color}
%\usepackage{color}
\usepackage{colortbl}
% выделение текста
\usepackage[usenames,dvipsnames,svgnames,table,rgb]{xcolor}
%\usepackage{xcolor}
\usepackage{varwidth}
\usepackage{lipsum}
% deprecated вариант
%\usepackage[normalem]{ulem}

%% Изображения, графика
\usepackage{graphicx}
% обтекание текстом
\usepackage{wrapfig}
% рисование
\usepackage{tikz}
\usepackage{pgfplots}
\usepackage{pgfplotstable}
\usetikzlibrary{positioning,shapes,shadows,arrows,intersections,patterns}
\pgfplotsset{width=7cm,compat=1.10}
\usepgfplotslibrary{fillbetween}

%% Таблицы
% дополнительные возможности
\usepackage{array,tabularx,tabulary,booktabs}
% таблицы на несколько страниц
\usepackage{longtable}
% форматирование таблиц (слияние строк, ячеек и др.)
\usepackage{multirow,makecell}

%% Библиография
% ссылки на литературу
\usepackage{cite}
% ссылки в верхних индексах
%\usepackage[superscript]{cite}
%\usepackage[nocompress]{cite}
% еще инструменты для ссылок
\usepackage{csquotes}
% biblatex [deprecated!]
%\usepackage[backend=biber,bibencoding=utf8,sorting=ynt,maxcitenames=2,style=authoryear]{biblatex}
%\usepackage[...,sorting=nty,...]{biblatex}
%\addbibresource{parts/biblio.bib}

%% Гиперссылки
\usepackage[linktocpage=true,plainpages=false,pdfpagelabels=false]{hyperref}

%% Оглавление
\usepackage{tocloft}

%% Программирование
% логические операторы
\usepackage{etoolbox}

%% Другие пакеты
% подсчет количества страниц в документе
\usepackage{lastpage}
% несколько колонок
\usepackage{multicol}
% счетчики для глав, приложений и т.д. [deprecated!?]
\usepackage{totcount}
% листинги исходных кодов
\usepackage{listings}

%% Дополнительно
% форматирование сносок
\usepackage{footmisc}
% ужимание текста
%\usepackage{savetrees}
% печать брошюрой (не тестировалось)
%\usepackage[print]{booklet}
% устранение "Font shape undefined" [http://ctan.org/pkg/lm,
% http://www.tex.ac.uk/cgi-bin/texfaq2html?label=fontsize]
%\usepackage{lmodern}
