%%% Колонтитулы (см. packages.tex) %%%
\pagestyle{fancy}
\renewcommand{\headrulewidth}{0pt}	% Толщина линейки, отчеркивающей верхний колонтитул
%\lfoot{Нижний левый}
%\rfoot{Нижний правый}
%\rhead{Верхний правый}
%\chead{Верхний в центре}
%\lhead{Верхний левый}
%\cfoot{Нижний в центре}			% По умолчанию здесь номер страницы
\lfoot{}
\rfoot{}
\rhead{}
\chead{\normalsize\thepage}
\lhead{}
\cfoot{}
% при использовании chapters - первая страница создается в plain pagestyle
\fancypagestyle{plain}{\pagestyle{fancy}}



%%% Интерлиньяж %%%
\onehalfspacing	% Интерлиньяж 1.5
%\doublespacing	% Интерлиньяж 2
%\singlespacing	% Интерлиньяж 1



%%% Макет страницы %%%
\geometry{a4paper,top=20mm,bottom=20mm,left=25mm,right=10mm}
% Абзацный отступ 5 знаков (5 пробелов ~ 2em)
\parindent=2em

% Заголовки от текста сверху и снизу - 3 интервала (по умолчанию)
% Частично рабочие варианты:
%	\usepackage{titlesec}
%	\setlength{\parskip}{0.5cm}
%	\titlespacing{\section}{0pt}{\parskip}{-\parskip}
%	\titlespacing{\subsection}{0pt}{\parskip}{-\parskip}
%	\titlespacing{\subsubsection}{0pt}{\parskip}{-\parskip}

\usepackage{sectsty}
% центрирование заголовков + запрет переносов (\hyphenation для всех)
%\allsectionsfont{\centering\raggedright}
\allsectionsfont{\centering\normalsize\raggedright}

% singlespacing в заголовках %
% Необходимо использовать \begin{singlespace}...\end{singlespace} если заголовок длинный
% Частично рабочие варианты:
%	\pretocmd{\@chap}{\singlespacing}{}{}
%	\apptocmd{\@chap}{\onehalfspacing}{}{}
%
%	\let\oldchap\chapter
%	\renewcommand*{\chapter}[1]{\singlespace\oldchap{#1}\onehalfspacing}
%
%	\usepackage{titlesec}
%	\titleformat{\chapter}
%	{\singlespacing\normalfont\Huge\bfseries}{\thechapter}{1em}{}

% центрирование глав + отступы после глав, добавление "." после номера разделов
%\sectionfont{\normalsize}
\usepackage{titlesec}
%\titleformat{\chapter}{\centering\normalsize\bf}{\thechapter.}{0pt}{\normalsize\bf}
%\titleformat{\chapter}{\centering\bf}{\thechapter.}{10pt}{\bf}
\titleformat{\chapter}{\centering\bf}{\thechapter.}{10pt}{\bf}
\titlespacing*{\chapter}{0pt}{*0}{2ex}
\titlelabel{\thetitle.\quad}

% Изменение размера заголовка Содержания
%\renewcommand{\cfttoctitlefont}{\hfill\Large\bfseries}
\renewcommand{\cfttoctitlefont}{\Large\bfseries}
% Точка после номеров разделов
\renewcommand{\cftchapaftersnum}{.}
\renewcommand{\cftsecaftersnum}{.}
\renewcommand{\cftsubsecaftersnum}{.}

% Изменение размера Списка рисунков и таблиц
\renewcommand\cftloftitlefont{\hspace*{\fill}\normalsize\bf}
\renewcommand{\cftafterloftitle}{\hspace*{\fill}}
\renewcommand\cftlottitlefont{\hspace*{\fill}\normalsize\bf}
\renewcommand{\cftafterlottitle}{\hspace*{\fill}}

%%% список сокращений %%%
\usepackage{nomencl}
% Позволяет одновременно печатать условное обозначение в тексте документа и добавлять его в перечень
\newcommand*{\nom}[2]{#1\nomenclature{#1}{#2}}
% Перечень сокращений будет распечатываться по алфавиту вне зависимости от появления в тексте
% Пример:
%	\nom{Б}{Вторая буква алфавита}.
%	А\nomenclature{А}{Первая буква алфавита}
\setlength\nomlabelwidth{5cm}



%%% Формат формул и ссылок на формулы %%%
% ПРИМЕР:
%\begin{equation}\label{name}
%	2+2=4
%\end{equation}
%
%Формула \eqref{name}



%%% Прочее: %%%
%	- Список литературы формируется в порядке возрастания - bib1, bib2, bib3,...
% 	- по ГОСТ НЕОБХОДИМО, чтобы источники на отличном от русского языке перечислялись ПОСЛЕ источников на русском
%	- Приложения ДОЛЖНЫ быть перечислены в порядке их перечисления в тексте

\def\labelitemi{--}


%%% отключение отступов у списков %%%
\usepackage{paralist}
\let\itemize\compactitem
\let\enditemize\endcompactitem
\let\enumerate\compactenum
\let\endenumerate\endcompactenum
\let\description\compactdesc
\let\enddescription\endcompactdesc
%\pltopsep=\medskipamount
%\plitemsep=1pt
%\plparsep=1pt

% некорректно работает с synopsis.tex
%\usepackage{enumitem}
%%\setlist[enumerate]{nosep}
%%\setlist[itemize]{nosep}
%%\setlist[description]{nosep}
%%\setlist{nolistsep}
%\setlist{nosep}


%%% подписи без ":" %%%
%%\DeclareCaptionLabelFormat{dash}{#1--#2}
%%\captionsetup[figure]{labelformat=dash}
%\captionsetup[figure]{labelsep=endash}
%\captionsetup[table]{labelsep=endash}
\captionsetup{labelsep=endash}

% special enumerate list with indentation, similar to enumitem-package
% "\setenumerate{listparindent=\parindent}"
\newenvironment{myenumerate}{%
  \edef\backupindent{\the\parindent}%
  \enumerate%
  \setlength{\parindent}{\backupindent}%
}{\endenumerate}


%%% ниже перенесенные стили из dissertation.tex/synopsis.tex

% сужаем литературу для сохранения места
%\usepackage{etoolbox}
\patchcmd\thebibliography
{\labelsep}
{\labelsep\itemsep=0pt\parsep=0pt\setlength{\leftmargin}{15pt}\relax}
{}
{\typeout{Couldn't patch the command}}

% для формирование отступов в enumerate и itemize (ниже)
\usepackage{\template/contrib/tweaklist}

% переопределяем установки для нумерованных перечней
\renewcommand{\enumhook}{
	% 1-й абзацный отступ
	\setlength{\itemindent}{1.7\parindent}
	% разделитель элементов
	\setlength{\itemsep}{0mm}
	% расстояние между маркером и текстом (0,5 кегля)
	\setlength{\labelsep}{.5em}
	% ширина маркера
	\setlength{\labelwidth}{0mm}
	% 2-й и последующие абзацные отступы
	\setlength{\listparindent}{1.5\parindent}
	% отступ списка слева
	\setlength{\leftmargin}{0mm}
	% отступ списка справа
	\setlength{\rightmargin}{0mm}
	% расстояние между элементами или абзацами
	\setlength{\parsep}{0mm}
	% отступ над вложенным списком
	\setlength{\parskip}{\parskip}
	% отступ списка
	\setlength{\partopsep}{0mm}
	% отступ вложенного списка
	\setlength{\topsep}{0mm}
}

% копируем настройки для перечней разного уровня
\renewcommand{\enumhooki}{\enumhook}
\renewcommand{\itemhooki}{\enumhook}
\renewcommand{\enumhookii}{\enumhook}
\renewcommand{\itemhookii}{\enumhook}

% введем счетчики списка литературы, # глав и приложений [deprecated!]
\newtotcounter{bibnum}
\newtotcounter{chapnum}
\newtotcounter{appendnum}
% сохраненный object
\def\oldcite{}
\def\oldchap{}
\def\oldappnd{}
\let\oldcite=\bibcite
\let\oldchap=\chapter
\let\oldappnd=\chapter
% object = object + увеличение счетчика %
\def\bibcite{\stepcounter{bibnum}\oldcite}
\def\realchapter{\stepcounter{chapnum}\oldchap}
\def\appndchapter{\stepcounter{appendnum}\oldappnd}

% добавим файл с информацией по работе, автору и т.д.
% !TEX root = dissertation.tex synopsis.tex

%
% phdtex
%
% Copyright (c) 2014-2020, Andrew Kanner <andrew.kanner@gmail.com>.
% All rights reserved.
%
% SPDX-License-Identifier: CC-BY-4.0
%

% заголовок
\author{ФАМИЛИЯ ИМЯ ОТЧЕСТВО автора}
\title{НАЗВАНИЕ ДИССЕРТАЦИОННОЙ РАБОТЫ}

% неизменяемые общие данные
\date{\today}
\makeatletter
\def\dissauthor{\@author}
\def\disstitle{\@title}
\def\dissdate{\@date}
\makeatother
\def\dissyear{\the\year}

% специальность, ученая степень, руководитель, консультант, оппоненты
\def\specnum{ХХ.ХХ.ХХ}
\def\specname{<<Название специальности>>}
\def\edudegree{кандидата каких-то там наук}
\def\mentordegree{уч. степень, уч. звание}
%\def\mentortitle{уч. звание}
\def\mentorcompany{Название\\компании,\\г. Город\\~}
\def\mentorname{Фамилия И. О.}
\def\consultantdegree{уч. степень}
\def\consultantcompany{Название компании,\\ г. Город}
\def\consultantname{Фамилия И. О.}
\def\opponentonedegree{уч. степень}
\def\opponentonecompany{\\Название компании,\\ г. Город}
\def\opponentonename{Фамилия И. О.}
\def\opponenttwodegree{уч. степень}
\def\opponenttwocompany{\\Название компании,\\ г. Город}
\def\opponenttwoname{Фамилия И. О.}
\def\opponentthreedegree{уч. степень}
\def\opponentthreecompany{\\Название компании,\\ г. Город}
\def\opponentthreename{Фамилия И. О.}
\def\leadingorg{Название ведущей организации}

% прочие данные
\def\disscity{Город1 }
\def\disscouncilcity{Город2 }
\def\dissorg{НАЗВАНИЕ УЧРЕЖДЕНИЯ, В КОТОРОМ ВЫПОЛНЯЛАСЬ\par
ДАННАЯ ДИССЕРТАЦИОННАЯ РАБОТА\par}
\def\disscouncil{НАЗВАНИЕ УЧРЕЖДЕНИЯ, В КОТОРОМ ЗАЩИЩАЛАСЬ\par
ДАННАЯ ДИССЕРТАЦИОННАЯ РАБОТА\par}
\def\dissorgsyn{<<Название учреждения, в котором выполнялась данная
  диссертационная работа>>}
\def\dissudk{УДК xxx.xxx}

% данные для оборота титульного листа автореферата
\def\councildate{<<\dots>> \dots \the\year~г.}
\def\counciltime{\dots:00}
\def\councilnum{Д xxx.xxx.xx}
\def\councilplace{<<Организация, к которой относится диссертационный
  совет>>}
\def\counciladdress{Индекс, г.~Город, шоссе / улица / \dots,~д.xx}
\def\libraryname{<<Организация, куда передается рукопись
  диссертационной работы>> и на сайте:}
\def\libraryaddress{Индекс, г.~Город, шоссе / улица / \dots,~д.xx}
\def\councilwebsite{http://website.ru}
\def\synopsissentdate{<<\dots>> \dots \the\year~года}
\def\councilsecretarydegree{уч. степень, уч. звание}
\def\councilsecretaryname{И.О.~Фамилия}

% выставим атрибуты pdf-документа
\hypersetup{
  % заголовок
  pdftitle={\disstitle},
  % автор
  pdfauthor={\dissauthor},
  % тема
  pdfsubject={\disstitle},
  % создатель
  pdfcreator={\dissauthor},
  % производитель
  pdfproducer={\dissauthor},
  % ключевые слова
  pdfkeywords={keyword1} {keyword2}
}


% неизменяемые общие данные
\date{\today}
\makeatletter
\def\dissauthor{\@author}
\def\disstitle{\@title}
\def\dissdate{\@date}
\makeatother
\def\dissyear{\the\year}

% выставим атрибуты pdf-документа
\hypersetup{
	% заголовок
	pdftitle={\disstitle},
	% автор
	pdfauthor={\dissauthor},
	% тема
	pdfsubject={\disstitle},
	% создатель
	pdfcreator={\dissauthor},
	% производитель
	pdfproducer={\dissauthor},
}

% для переопределения параметров geometry (для правильной работы fancyhdr)
\makeatletter
\newcommand{\resetHeadWidth}{\fancy@setoffs}
\makeatother