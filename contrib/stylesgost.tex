%
% phdtex
%
% Copyright (c) 2014-2020, Andrew Kanner <andrew.kanner@gmail.com>.
% All rights reserved.
%
% SPDX-License-Identifier: MIT
%

%% Макет страницы, поля и разметка
\geometry{a4paper,top=20mm,bottom=20mm,left=25mm,right=10mm}

%% Оформление абзацев, колонтитулов, текста
% колонтитулы
\pagestyle{fancy}
% толщина линейки, отчеркивающей верхний колонтитул
\renewcommand{\headrulewidth}{0pt}
% нижний левый
\lfoot{}
% нижний правый
\rfoot{}
% верхний правый
\rhead{}
% верхний в центре, номер страницы здесь
\chead{\normalsize\thepage}
% верхний левый
\lhead{}
% нижний в центре
\cfoot{}

% текст
% при использовании chapters первая страница в plain pagestyle
\fancypagestyle{plain}{\pagestyle{fancy}}
% интерлиньяж 1.5 (либо \singlespacing, \doublespacing)
\onehalfspacing
% абзацный отступ 5 знаков (5 пробелов ~ 2em)
\parindent=2em

% для переопределения параметров geometry (правильной работы fancyhdr)
\makeatletter
\newcommand{\resetHeadWidth}{\f@nch@setoffs}
\makeatother

% заголовки (от текста сверху и снизу 3 интервала, по умолчанию)
% deprecated вариант
%\usepackage{titlesec}
%\setlength{\parskip}{0.5cm}
%\titlespacing{\section}{0pt}{\parskip}{-\parskip}
%\titlespacing{\subsection}{0pt}{\parskip}{-\parskip}
%\titlespacing{\subsubsection}{0pt}{\parskip}{-\parskip}

% нумеровать разделы до subsubsection включительно (в report только subsection)
\setcounter{secnumdepth}{3}

% центрирование заголовков, запрет переносов (\hyphenation для всех)
\usepackage{sectsty}
\allsectionsfont{\centering\normalsize\raggedright}

% центрирование глав, отступы после, "." после номеров разделов
\usepackage{titlesec}
%\sectionfont{\normalsize}
\titleformat{\chapter}{\centering\bf}{\thechapter.}{10pt}{\bf}
\titlespacing*{\chapter}{0pt}{*0}{2ex}
\titlelabel{\thetitle.\quad}

% изменение размера заголовка Оглавление, центрирование
\renewcommand{\cfttoctitlefont}{\hfill\large\bfseries}
\renewcommand{\cftaftertoctitle}{\hfill}
% "." после номеров разделов
\renewcommand{\cftchapaftersnum}{.}
\renewcommand{\cftsecaftersnum}{.}
\renewcommand{\cftsubsecaftersnum}{.}
\setlength\cftbeforetoctitleskip{0pt}
\setlength\cftaftertoctitleskip{0pt}

% singlespacing в заголовках (вставлено в основной текст)
% deprecated варианты:
%\pretocmd{\@chap}{\singlespacing}{}{}
%\apptocmd{\@chap}{\onehalfspacing}{}{}
%%\let\oldchap\chapter
%%\renewcommand*{\chapter}[1]{\singlespace\oldchap{#1}\onehalfspacing}
%%%\usepackage{titlesec}
%%%\titleformat{\chapter}
%%%{\singlespacing\normalfont\Huge\bfseries}{\thechapter}{1em}{}

%% Списки
% переопределение символа ненумерованного списка
\def\labelitemi{--}
% отключение отступов у списков
\usepackage{paralist}
\let\itemize\compactitem
\let\enditemize\endcompactitem
\let\enumerate\compactenum
\let\endenumerate\endcompactenum
\let\description\compactdesc
\let\enddescription\endcompactdesc
%\pltopsep=\medskipamount
%\plitemsep=1pt
%\plparsep=1pt
% deprecated вариант
%\usepackage{enumitem}
%%\setlist[enumerate]{nosep}
%%\setlist[itemize]{nosep}
%%\setlist[description]{nosep}
%%\setlist{nolistsep}
%\setlist{nosep}

% для формирование отступов в enumerate и itemize
\usepackage{\template/contrib/tweaklist}
% переопределяем установки для нумерованных перечней
\renewcommand{\enumhook}{
  % 1-й абзацный отступ
  \setlength{\itemindent}{1.7\parindent}
  % разделитель элементов
  \setlength{\itemsep}{0mm}
  % расстояние между маркером и текстом (0,5 кегля)
  \setlength{\labelsep}{.5em}
  % ширина маркера
  \setlength{\labelwidth}{0mm}
  % 2-й и последующие абзацные отступы
  \setlength{\listparindent}{1.5\parindent}
  % отступ списка слева
  \setlength{\leftmargin}{0mm}
  % отступ списка справа
  \setlength{\rightmargin}{0mm}
  % расстояние между элементами или абзацами
  \setlength{\parsep}{0mm}
  % отступ над вложенным списком
  \setlength{\parskip}{\parskip}
  % отступ списка
  \setlength{\partopsep}{0mm}
  % отступ вложенного списка
  \setlength{\topsep}{0mm}
}
% копируем настройки для перечней разного уровня
\renewcommand{\enumhooki}{\enumhook}
\renewcommand{\itemhooki}{\enumhook}
\renewcommand{\enumhookii}{\enumhook}
\renewcommand{\itemhookii}{\enumhook}

% специальный список с индентацией (аналогично пакету enumitem)
\newenvironment{myenumerate}{%
  \edef\backupindent{\the\parindent}%
  \enumerate%
  \setlength{\parindent}{\backupindent}%
}{\endenumerate}

%% Список рисунков и таблиц (изменение размера, отступов)
\renewcommand\cftloftitlefont{\hspace*{\fill}\normalsize}
\renewcommand{\cftafterloftitle}{\hspace*{\fill}}
\renewcommand\cftlottitlefont{\hspace*{\fill}\normalsize}
\renewcommand{\cftafterlottitle}{\hspace*{\fill}}
\setlength\cftbeforeloftitleskip{0pt}
\setlength\cftafterloftitleskip{0pt}
\setlength\cftbeforelottitleskip{0pt}
\setlength\cftafterlottitleskip{0pt}

%% Список сокращений
\usepackage{nomencl}
% печатает условное обозначение в тексте и добавлять его в перечень
\newcommand*{\nom}[2]{#1\nomenclature{#1}{#2}}
\setlength\nomlabelwidth{5cm}

%% Список литературы
% сужаем литературу для сохранения места
%\usepackage{etoolbox}
\patchcmd\thebibliography
{\labelsep}
{\labelsep\itemsep=0pt\parsep=0pt\setlength{\leftmargin}{15pt}\relax}
{}
{\typeout{Couldn't patch the command}}

%% Подписи к объектам (без ":")
\captionsetup{labelsep=endash}
%\captionsetup[figure]{labelsep=endash}
%\captionsetup[table]{labelsep=endash}

%% Счетчики списка литературы, # глав и приложений [deprecated!].
%% В тексте можно заменить окружения realchapter/appndchapter на
%% chapter, т.к. они использовались для правильного подсчитывания
%% глав, приложений и т.п. (сейчас нужно только в автореферате).
\newtotcounter{bibnum}
\newtotcounter{chapnum}
\newtotcounter{appendnum}
\newtotcounter{fignum}
\newtotcounter{tabnum}
% сохраненный object
\def\oldcite{}
\def\oldchap{}
\def\oldappnd{}
\let\oldcite=\bibcite
\let\oldchap=\chapter
\let\oldappnd=\chapter
% object = object + увеличение счетчика %
\def\bibcite{\stepcounter{bibnum}\oldcite}
% счетчики рисунков и таблиц сбрасываются в главах -- просуммируем
\def\realchapter{\stepcounter{chapnum}\addtocounter{fignum}{\value{figure}}\addtocounter{tabnum}{\value{table}}\oldchap}
\def\appndchapter{\stepcounter{appendnum}\oldappnd}
