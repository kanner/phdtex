%%% Колонтитулы (см. packages.tex) %%%
\pagestyle{fancy}
\renewcommand{\headrulewidth}{0pt}	% Толщина линейки, отчеркивающей верхний колонтитул
%\lfoot{Нижний левый}
%\rfoot{Нижний правый}
%\rhead{Верхний правый}
%\chead{Верхний в центре}
%\lhead{Верхний левый}
%\cfoot{Нижний в центре}			% По умолчанию здесь номер страницы
\lfoot{}
\rfoot{}
\rhead{}
\chead{\normalsize\thepage}
\lhead{}
\cfoot{}

%%% при использовании chapters - первая страница создается в plain pagestyle %%%
\fancypagestyle{plain}{\pagestyle{fancy}}

%%% Интерлиньяж %%%
\onehalfspacing	% Интерлиньяж 1.5
%\doublespacing	% Интерлиньяж 2
%\singlespacing	% Интерлиньяж 1

%%% Макет страницы %%%
\geometry{a4paper,top=20mm,bottom=20mm,left=25mm,right=10mm}

\usepackage{sectsty}
\allsectionsfont{\centering}	% центрирование заголовков (должны быть без точки на конце и переносов)

%%% список сокращений %%%
\usepackage{nomencl}

% Позволяет одновременно печатать условное обозначение в тексте документа и добавлять его в перечень
\newcommand*{\nom}[2]{#1\nomenclature{#1}{#2}}

%%% Формат формул и ссылок на формулы %%%
% ПРИМЕР:
%\begin{equation}\label{name}
%	2+2=4
%\end{equation}
%
%Формула \eqref{name}

% Перечень сокращений будет распечатываться по алфавиту вне зависимости от появления в тексте
% ПРИМЕР:
%\nom{Б}{Вторая буква алфавита}.
%А\nomenclature{А}{Первая буква алфавита}

% Список литературы формируется в порядке возрастания - bib1, bib2, bib3,...
% по ГОСТ НЕОБХОДИМО, чтобы источники на отличном от русского языке перечислялись ПОСЛЕ источников на русском

% Приложения ДОЛЖНЫ быть перечислены в порядке их перечисления в тексте