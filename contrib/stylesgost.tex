%%% Колонтитулы (см. packages.tex) %%%
\pagestyle{fancy}
\renewcommand{\headrulewidth}{0pt}	% Толщина линейки, отчеркивающей верхний колонтитул
%\lfoot{Нижний левый}
%\rfoot{Нижний правый}
%\rhead{Верхний правый}
%\chead{Верхний в центре}
%\lhead{Верхний левый}
%\cfoot{Нижний в центре}			% По умолчанию здесь номер страницы
\lfoot{}
\rfoot{}
\rhead{}
\chead{\normalsize\thepage}
\lhead{}
\cfoot{}
% при использовании chapters - первая страница создается в plain pagestyle
\fancypagestyle{plain}{\pagestyle{fancy}}



%%% Интерлиньяж %%%
\onehalfspacing	% Интерлиньяж 1.5
%\doublespacing	% Интерлиньяж 2
%\singlespacing	% Интерлиньяж 1



%%% Макет страницы %%%
\geometry{a4paper,top=20mm,bottom=20mm,left=25mm,right=10mm}
% Абзацный отступ 5 знаков (5 пробелов ~ 2em)
\parindent=2em

% Заголовки от текста сверху и снизу - 3 интервала (по умолчанию)
% Частично рабочие варианты:
%	\usepackage{titlesec}
%	\setlength{\parskip}{0.5cm}
%	\titlespacing{\section}{0pt}{\parskip}{-\parskip}
%	\titlespacing{\subsection}{0pt}{\parskip}{-\parskip}
%	\titlespacing{\subsubsection}{0pt}{\parskip}{-\parskip}

\usepackage{sectsty}
% центрирование заголовков + запрет переносов (\hyphenation для всех)
%\allsectionsfont{\centering\raggedright}
\allsectionsfont{\centering\normalsize\raggedright}

% singlespacing в заголовках %
% Необходимо использовать \begin{singlespace}...\end{singlespace} если заголовок длинный
% Частично рабочие варианты:
%	\pretocmd{\@chap}{\singlespacing}{}{}
%	\apptocmd{\@chap}{\onehalfspacing}{}{}
%
%	\let\oldchap\chapter
%	\renewcommand*{\chapter}[1]{\singlespace\oldchap{#1}\onehalfspacing}
%
%	\usepackage{titlesec}
%	\titleformat{\chapter}
%	{\singlespacing\normalfont\Huge\bfseries}{\thechapter}{1em}{}

% центрирование глав + отступы после глав, добавление "." после номера разделов
%\sectionfont{\normalsize}
\usepackage{titlesec}
%\titleformat{\chapter}{\centering\normalsize\bf}{\thechapter.}{0pt}{\normalsize\bf}
%\titleformat{\chapter}{\centering\bf}{\thechapter.}{10pt}{\bf}
\titleformat{\chapter}{\centering\bf}{\thechapter.}{10pt}{\bf}
\titlespacing*{\chapter}{0pt}{*0}{2ex}
\titlelabel{\thetitle.\quad}

% Изменение размера заголовка Содержания
%\renewcommand{\cfttoctitlefont}{\hfill\Large\bfseries}
\renewcommand{\cfttoctitlefont}{\Large\bfseries}
% Точка после номеров разделов
\renewcommand{\cftchapaftersnum}{.}
\renewcommand{\cftsecaftersnum}{.}
\renewcommand{\cftsubsecaftersnum}{.}

% Изменение размера Списка рисунков и таблиц
\renewcommand\cftloftitlefont{\hspace*{\fill}\normalsize\bf}
\renewcommand{\cftafterloftitle}{\hspace*{\fill}}
\renewcommand\cftlottitlefont{\hspace*{\fill}\normalsize\bf}
\renewcommand{\cftafterlottitle}{\hspace*{\fill}}

%%% список сокращений %%%
\usepackage{nomencl}
% Позволяет одновременно печатать условное обозначение в тексте документа и добавлять его в перечень
\newcommand*{\nom}[2]{#1\nomenclature{#1}{#2}}
% Перечень сокращений будет распечатываться по алфавиту вне зависимости от появления в тексте
% Пример:
%	\nom{Б}{Вторая буква алфавита}.
%	А\nomenclature{А}{Первая буква алфавита}
\setlength\nomlabelwidth{5cm}



%%% Формат формул и ссылок на формулы %%%
% ПРИМЕР:
%\begin{equation}\label{name}
%	2+2=4
%\end{equation}
%
%Формула \eqref{name}



%%% Прочее: %%%
%	- Список литературы формируется в порядке возрастания - bib1, bib2, bib3,...
% 	- по ГОСТ НЕОБХОДИМО, чтобы источники на отличном от русского языке перечислялись ПОСЛЕ источников на русском
%	- Приложения ДОЛЖНЫ быть перечислены в порядке их перечисления в тексте

\def\labelitemi{--}


%%% отключение отступов у списков %%%
\usepackage{paralist}
\let\itemize\compactitem
\let\enditemize\endcompactitem
\let\enumerate\compactenum
\let\endenumerate\endcompactenum
\let\description\compactdesc
\let\enddescription\endcompactdesc
%\pltopsep=\medskipamount
%\plitemsep=1pt
%\plparsep=1pt

% некорректно работает с synopsis.tex
%\usepackage{enumitem}
%%\setlist[enumerate]{nosep}
%%\setlist[itemize]{nosep}
%%\setlist[description]{nosep}
%%\setlist{nolistsep}
%\setlist{nosep}


%%% подписи без ":" %%%
%%\DeclareCaptionLabelFormat{dash}{#1--#2}
%%\captionsetup[figure]{labelformat=dash}
%\captionsetup[figure]{labelsep=endash}
%\captionsetup[table]{labelsep=endash}
\captionsetup{labelsep=endash}

% special enumerate list with indentation, similar to enumitem-package
% "\setenumerate{listparindent=\parindent}"
\newenvironment{myenumerate}{%
  \edef\backupindent{\the\parindent}%
  \enumerate%
  \setlength{\parindent}{\backupindent}%
}{\endenumerate}
