% переопределение именований
\renewcommand{\abstractname}{Аннотация}
\renewcommand{\alsoname}{см. также}
\renewcommand{\appendixname}{Приложение}
\renewcommand{\bibname}{Список использованных источников}
\renewcommand{\ccname}{исх.}
\renewcommand{\chaptername}{Глава}
\renewcommand{\contentsname}{Содержание}
\renewcommand{\enclname}{вкл.}
\renewcommand{\figurename}{Рисунок}
\renewcommand{\headtoname}{вх.}
\renewcommand{\indexname}{Предметный указатель}
\renewcommand{\listfigurename}{Список рисунков}
\renewcommand{\listtablename}{Список таблиц}
\renewcommand{\pagename}{Стр.}
\renewcommand{\partname}{Часть}
\renewcommand{\refname}{Список литературы}
\renewcommand{\seename}{см.}
\renewcommand{\tablename}{Таблица}
\renewcommand{\lstlistingname}{Листинг}

% заголовок перечня nomenclature
\renewcommand{\nomname}{Перечень сокращений и условных обозначений}

% переопределение математических символов на русский манер
\renewcommand{\epsilon}{\ensuremath{\varepsilon}}
\renewcommand{\phi}{\ensuremath{\varphi}}
\renewcommand{\kappa}{\ensuremath{\varkappa}}
\renewcommand{\le}{\ensuremath{\leqslant}}
\renewcommand{\leq}{\ensuremath{\leqslant}}
\renewcommand{\ge}{\ensuremath{\geqslant}}
\renewcommand{\geq}{\ensuremath{\geqslant}}
\renewcommand{\emptyset}{\varnothing}

% перенос знаков в формулах (по Львовскому)
\newcommand*{\hm}[1]{#1\nobreak\discretionary{}
{\hbox{$\mathsurround=0pt #1$}}{}}

% сокращения
\newcommand*{\linux}{GNU/Linux}
\newcommand*{\linuxkernel}{Linux}

% дополнительные команды
%\DeclareMathOperator{\sgn}{\mathop{sgn}}
\newcommand*{\nbline}{\\*\indent}
\newcommand{\tab}[1]{\hspace{.1\textwidth}\rlap{#1}}
\newcommand*{\ti}[1]{\text{\textit{#1}}}

% счетчик для пунктов в первом столбце таблиц
\newcounter{rowcount}
\setcounter{rowcount}{0}
\newcommand\rownumber{\stepcounter{rowcount}\arabic{rowcount}.}
