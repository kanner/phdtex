%%% Колонтитулы (см. packages.tex) %%%
\pagestyle{fancy}
\renewcommand{\headrulewidth}{0pt}	% Толщина линейки, отчеркивающей верхний колонтитул
%\lfoot{Нижний левый}
%\rfoot{Нижний правый}
%\rhead{Верхний правый}
%\chead{Верхний в центре}
%\lhead{Верхний левый}
%\cfoot{Нижний в центре}			% По умолчанию здесь номер страницы
\lfoot{}
\rfoot{}
\rhead{}
\chead{\normalsize\thepage}
\lhead{}
\cfoot{}

%%% при использовании chapters - первая страница создается в plain pagestyle %%%
\fancypagestyle{plain}{\pagestyle{fancy}}

%%% Интерлиньяж %%%
\onehalfspacing	% Интерлиньяж 1.5
%\doublespacing	% Интерлиньяж 2
%\singlespacing	% Интерлиньяж 1

%%% Макет страницы %%%
\geometry{a4paper,top=20mm,bottom=20mm,left=25mm,right=10mm}

%%% Кодировки и шрифты %%%
\renewcommand{\rmdefault}{ftm} % Включаем Times New Roman

%%% Альтернативные шрифты (см. packages.tex) %%%
%\defaultfontfeatures{Ligatures={TeX},Renderer=Basic}		% свойства шрифтов по умолчанию
%\setmainfont[Ligatures={TeX,Historic}]{Times New Roman}	% основной шрифт документа
%\setsansfont{Comic Sans MS}								% шрифт без засечек
%\setmonofont{Courier New}
%\renewcommand{\familydefault}{\sfdefault}					% Начертание шрифта

%%% Номера формул %%%
%\mathtoolsset{showonlyrefs=true}	% Показывать номера только у тех формул, на которые есть \eqref{} в тексте.

%%% Выравнивание и переносы %%%
\sloppy					% Избавляемся от переполнений
\clubpenalty=10000		% Запрещаем разрыв страницы после первой строки абзаца
\widowpenalty=10000		% Запрещаем разрыв страницы после последней строки абзаца

%%% Библиография %%%
\makeatletter
\bibliographystyle{./contrib/utf8gost705u}	% Оформляем библиографию в соответствии с ГОСТ 7.0.5
\renewcommand{\@biblabel}[1]{#1.}	% Заменяем библиографию с квадратных скобок на точку:
\makeatother

%%% Изображения %%%
\graphicspath{{images/}}	% Пути к изображениям
\setlength\fboxsep{3pt}		% Отступ рамки \fbox{} от рисунка
\setlength\fboxrule{1pt}	% Толщина линий рамки \fbox{}

%%% Цвета гиперссылок %%%
\definecolor{linkcolor}{rgb}{0.9,0,0}
\definecolor{citecolor}{rgb}{0,0.6,0}
\definecolor{urlcolor}{rgb}{0,0,1}
\hypersetup{					% Гиперссылки
	unicode=true,				% русские буквы в раздела PDF
    colorlinks=true,			% false: ссылки в рамках; true: цветные ссылки
    linkcolor={linkcolor},		% внутренние ссылки
    citecolor={citecolor},		% на библиографию
    filecolor=magenta,			% на файлы
    urlcolor={urlcolor}			% на URL
}

%%% Оглавление %%%
\renewcommand{\cftchapdotsep}{\cftdotsep}

%%% Теоремы %%%
\theoremstyle{plain}		% Это стиль по умолчанию, его можно не переопределять.
\newtheorem{theorem}{Теорема}[section]
\newtheorem{proposition}[theorem]{Утверждение}
 
\theoremstyle{definition}	% "Определение"
\newtheorem{corollary}{Следствие}[theorem]
\newtheorem{problem}{Задача}[section]
 
\theoremstyle{remark}		% "Примечание"
\newtheorem*{nonum}{Решение}
