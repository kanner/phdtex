%%% Макет страницы %%%
%\geometry{a4paper,top=2cm,bottom=2cm,left=3cm,right=1cm}

%%% Кодировки и шрифты %%%
\renewcommand{\rmdefault}{ftm} % Включаем Times New Roman

%%% Альтернативные шрифты (см. packages.tex) %%%
%\defaultfontfeatures{Ligatures={TeX},Renderer=Basic}		% свойства шрифтов по умолчанию
%\setmainfont[Ligatures={TeX,Historic}]{Times New Roman}	% основной шрифт документа
%\setsansfont{Comic Sans MS}								% шрифт без засечек
%\setmonofont{Courier New}
%\renewcommand{\familydefault}{\sfdefault}					% Начертание шрифта

%%% Номера формул %%%
%\mathtoolsset{showonlyrefs=true}	% Показывать номера только у тех формул, на которые есть \eqref{} в тексте.

%%% Выравнивание и переносы %%%
\sloppy					% Избавляемся от переполнений
\clubpenalty=10000		% Запрещаем разрыв страницы после первой строки абзаца
\widowpenalty=10000		% Запрещаем разрыв страницы после последней строки абзаца

%%% Библиография %%%
\makeatletter
\bibliographystyle{./contrib/utf8gost71my}	% Оформляем библиографию в соответствии с ГОСТ 7.0.5
\renewcommand{\@biblabel}[1]{#1.}	% Заменяем библиографию с квадратных скобок на точку:
\makeatother

%%% Изображения %%%
\graphicspath{{images/}}	% Пути к изображениям
\setlength\fboxsep{3pt}		% Отступ рамки \fbox{} от рисунка
\setlength\fboxrule{1pt}	% Толщина линий рамки \fbox{}

%%% Цвета гиперссылок %%%
\definecolor{linkcolor}{rgb}{0.9,0,0}
\definecolor{citecolor}{rgb}{0,0.6,0}
\definecolor{urlcolor}{rgb}{0,0,1}
\hypersetup{					% Гиперссылки
	unicode=true,				% русские буквы в раздела PDF
    colorlinks=true,			% false: ссылки в рамках; true: цветные ссылки
    linkcolor={linkcolor},		% внутренние ссылки
    citecolor={citecolor},		% на библиографию
    filecolor=magenta,			% на файлы
    urlcolor={urlcolor}			% на URL
}

%%% Оглавление %%%
\renewcommand{\cftchapdotsep}{\cftdotsep}

%%% Теоремы %%%
\theoremstyle{plain}		% Это стиль по умолчанию, его можно не переопределять.
\newtheorem{theorem}{Теорема}
\newtheorem{proposition}{Утверждение}
\newtheorem{lemma}{Лемма}

\theoremstyle{definition}	% "Определение"
\newtheorem{corollary}{Следствие}
\newtheorem{problem}{Задача}[section]
\newtheorem{condition}{Условие}
\newtheorem{definition}{Определение}
\newtheorem{axiom}{Аксиома}

\theoremstyle{remark}		% "Примечание"
\newtheorem*{nonum}{Решение}

%%% Выделение текста %%%
\newcommand\hly[1]{\colorbox{yellow!80}{\begin{varwidth}{\dimexpr\linewidth-2\fboxsep}#1\end{varwidth}}}
\newcommand\hlg[1]{\colorbox{green!80}{\begin{varwidth}{\dimexpr\linewidth-2\fboxsep}#1\end{varwidth}}}

%%% Цветные текст-боксы для таблиц и графиков %%%
\newcommand\hlyy[1]{\fcolorbox{black!50}{yellow!25}{\begin{varwidth}{\dimexpr\linewidth-2\fboxsep}#1\end{varwidth}}}
\newcommand\hlgg[1]{\fcolorbox{black!50}{green!25}{\begin{varwidth}{\dimexpr\linewidth-2\fboxsep}#1\end{varwidth}}}
\newcommand\hlbrr[1]{\fcolorbox{black!50}{brown!25}{\begin{varwidth}{\dimexpr\linewidth-2\fboxsep}#1\end{varwidth}}}
\newcommand\hlbll[1]{\fcolorbox{black!50}{black!25}{\begin{varwidth}{\dimexpr\linewidth-2\fboxsep}#1\end{varwidth}}}
\newcommand\hlrr[1]{\fcolorbox{black!50}{red!25}{\begin{varwidth}{\dimexpr\linewidth-2\fboxsep}#1\end{varwidth}}}
\newcommand\hlbluu[1]{\fcolorbox{black!50}{blue!25}{\begin{varwidth}{\dimexpr\linewidth-2\fboxsep}#1\end{varwidth}}}

\newcommand\hlempty[1]{\fcolorbox{black!50}{white}{\begin{varwidth}{\dimexpr\linewidth-2\fboxsep}#1\end{varwidth}}}

\newcommand\hlbr[1]{\colorbox{brown!80}{\begin{varwidth}{\dimexpr\linewidth-2\fboxsep}#1\end{varwidth}}}
\newcommand\hlbl[1]{\colorbox{black!80}{\begin{varwidth}{\dimexpr\linewidth-2\fboxsep}#1\end{varwidth}}}
\newcommand\hlr[1]{\colorbox{red!80}{\begin{varwidth}{\dimexpr\linewidth-2\fboxsep}#1\end{varwidth}}}
\newcommand\hlblu[1]{\colorbox{blue!80}{\begin{varwidth}{\dimexpr\linewidth-2\fboxsep}#1\end{varwidth}}}

% второй вариант
%\newcommand\hl{\bgroup\markoverwith %
%	{\textcolor{yellow}{\rule[-.5ex]{2pt}{2.5ex}}}\ULon}
% третий вариант
%\newcommand*{\hl}[1]{\colorbox{yellow}{#1}}

%%% Стили JavaScript %%%
\lstdefinelanguage{JavaScript}{
  keywords={break, case, catch, continue, debugger, default, delete, do, else, false, finally, for, function, if, in, instanceof, new, null, return, switch, this, throw, true, try, typeof, var, void, while, with},
  comment=[l]{//},
  morecomment=[s]{/*}{*/},
  morestring=[b]',
  morestring=[b]",
  ndkeywords={class, export, boolean, throw, implements, import, this},
  keywordstyle=\color{blue}\bfseries,
  ndkeywordstyle=\color{yellow}\bfseries,
  identifierstyle=\color{black},
  commentstyle=\color{green}\ttfamily,
  stringstyle=\color{red}\ttfamily
  sensitive=true,
}

%%% Стили XML %%%
\lstdefinelanguage{xml}{
	morekeywords={name,description,memory,unit,os,arch,machine,devices,emulator,hostdev, mode, type, managed, source, vendor, id, product, address, domain, bus, slot, function},
	numbers=none,
	frame=none,
	belowskip=2pt,
	caption=,
}

%%% Стили Bash %%%
\lstdefinelanguage{bashhh}{
	morekeywords={modprobe,echo},
	deletekeywords={bind},
	morecomment=[l]{\#},
	numbers=none,
	frame=none,
	belowskip=2pt,
	caption=,
}

%%% Цветовые схемы для выделения блоков кода %%%
\definecolor{codegreen}{rgb}{0,0.6,0}
\definecolor{codegray}{rgb}{0.5,0.5,0.5}
\definecolor{codemauve}{rgb}{0.58,0,0.82}
\lstset{
  columns=fixed,					% Делаем моноширинный шрифт
  backgroundcolor=\color{white},	% Цвет фона, нужно подключить пакет color или xcolor
  basicstyle=\small\sffamily,		% Размер и начертание
  breakatwhitespace=false,			% Переносим строки только при наличии пробела
  breaklines=true,					% Автоматически переносим строки
  captionpos=b,						% Позиция заголовка вверху [t] или внизу [b]
  commentstyle=\color{codegreen},	% Стиль для комментариев
  deletekeywords={...},				% Удалить какие-нибудь ключевые слова из языка
  escapeinside={\%*}{*)},			% Если вы хотите использовать LaTeX в вашем коде
  extendedchars=true,				% Позволяем использовать не-ASCII символы
%  extendedchars=\true,
% inputencoding=utf8x,
%  keepspaces = true,
  frame=single,						% Добавляем рамку вокруг кода
  keywordstyle=\color{blue},		% Стиль для ключевых слов
%  language=C,						% Язык программирования должен выставляться в самом листинге
%  language=JavaScript,
  morekeywords={*,...},				% Пользовательские ключевые слова
  numbers=left,						% Позиция номеров строк
  numbersep=5pt,					% Как далеко номера строк находятся от кода
  numberstyle=\tiny\color{codegray},% Стиль для номеров строк
  rulecolor=\color{black},			% Если не установлено, то цвет рамки может меняться
  showspaces=false,					% Показывать ли пробелы с помощью специальных отступов
  showstringspaces=false,			% Показывать ли пробелы в строках
  showtabs=false,					% Показывать ли знаки табуляции
  stepnumber=1,						% Размер шага между номерами строк
  stringstyle=\color{codemauve},	% Стиль для строковых литералов
  tabsize=2,						% Размер табуляции
  title=\lstname					% Показывать имя подключаемого файла
}

%%% для растягивания или игнорирования "Underfull \hbox"
%\setlength{\emergencystretch}{1pt}
%\sloppy
%\hyphenation{сло-во} % глобальные правила переноса (или запрета переноса)
%\hfuzz=2.5pt % 0.1pt -- стандартный допустимый "Overfull"
%\tolerance=400 % 200 -- стандартный порог разреженности

%%% использовать \slash -- для переноса слов, разделенных "\"?
